
Importantly, this does not mean that the domain model learning algorithm must output a model in the same representation as the input trajectories. 


while the learned model can be represented in a different representation.

while the input trajectories are given in a specific domain representation, the learned model can be represented in a different representation. 
The learned model is a planning domain model that can be given as input to a classical planner. 
Lastly, the 






Every action model learning algorithm can be viewed as a function that accepts a set of trajectories from the original domain and a learned domain. Therefore, we need to normalize the data of learned domains produced by different learning algorithms . 

To mitigate this, we introduce an \textbf{encoder-decoder} mechanism: The \textbf{decoder} maps the components (i.e. object types, fluents, action naming, signature ordering of object and st.) in the learned model \( m \) back to the original domain language \( m^* \). The  \textbf{encoder} transforms the original domain \( m^* \) into the representation format of the learned model \( m \). This process ensures that comparisons are done in a normalized format, reducing discrepancies caused by syntactic variations. Let us formally describe the encode and decode functions:
\begin{itemize}
    \item ${encodeAction: A_{m^*}\rightarrow P(A_m)}$$\text{ and} \\ encodeAction(s,a,m^*,m)$=
${\{a'\in m| a'\ is\ a\ proxy\ action\ of\ a\}}$
    \item ${\mathit{decodeAction}: A_{m}\rightarrow  A_{m^*} }$ $\text{ and } {\mathit{decodeAction}(s,a',m^*,m)}$$  a^*\in A_{m^*}$ s.t. $a^*$ is a proxy of $a$. 
\end{itemize}

\begin{itemize}
    \item $encodePlan: \{\pi_{m^*}\in \pi(A_{m^*})\}\rightarrow P(\pi_{m}\in \pi(A_{m}))$
    \item $\mathit{decodePlan}:\{\pi_{m}\in \pi(A_{m})\} \rightarrow  \{\pi_{m^*}\in \pi(A_{m^*})\}$
\end{itemize}

\begin{itemize}
    \item $encodePredicates:P( P_{m^*}) \rightarrow  P(P_m) \\$  and $encode{Predicates}(P'\subseteq P^*,m^*,m) = P''\subseteq P_m$
    \item $\mathit{decodePredicates}: P( P_{m}) \rightarrow  P(P_{m^*}) $  and $\mathit{decodePredicates}(P'\subseteq P_{m^*},m^*,m) = P''\subseteq P_m$. 
\end{itemize}

using these functions, we can now describe and use the earlier mentioned evaluation metrics to systematically compare the learned action model \( m \) with the ground truth model \( m^* \).

\subsubsection{semantic metric}
\paragraph{True Positives (TP):}
The set of instances where for every pre-state $s$ and action $a \in A_{m^*}$, if $a$ is applicable in $s$, then at least one learned action $a \in encode(s, a, m^*, m)$ is applicable in $s$.
\paragraph{False Positives (FP):}
The set of instances where there exists a learned action $a \in A_m$ that is applicable in a pre-state $s$, but there is no corresponding action $ a^* \in\ A_{m^*}\ $ $s.t.$ $a^*=decode(s,a',m^*,m) \ $ and $a^*$ is applicable in $s$.
\paragraph{False Negatives (FN):}
The set of instances where for some pre-state $s$ and action $a^* \in A_{m^*}$ $s.t.$ $a^*$ is applicable in $s$, but no learned action $a \in encode(s, a, m^*, m)$ is applicable in $s$.
\paragraph{True Negatives (TN):}
The set of instances where for every pre-state $s$, if a proxy action $a \in A_m$ is not applicable in $s$, then there is no corresponding action $a* \in A_{m^*}$ such that $a$ is applicable in $s$.

\begin{table}[h]
    \centering
    \begin{tabular}{|c|c|c|}
        \hline
        \textbf{Action Applicability} & \textbf{Original Model \( m^* \)} & \textbf{Learned Model \( m \)} \\
        \hline
        True Positive (TP) & Applicable & Applicable \\
        False Positive (FP) & Not Applicable & Applicable \\
        True Negative (TN) & Not Applicable & Not Applicable \\
        False Negative (FN) & Applicable & Not Applicable \\
        \hline
    \end{tabular}
    \caption{Comparison of action applicability in the original and learned models.}
    \label{tab:action-model-comparison}
\end{table}
These metrics allow us to quantify the \emph{precision} and \emph{recall}, where \emph{precision} is $\frac{TP}{TP+FP}$ 
and \emph{recall} is $\frac{TP}{TP+FN}$ This quantification will be useful when choosing the action model learning method based on restrictions and use cases of the learned model. These metrics provide a comprehensive evaluation of the learned model's ability to replicate the original model's behavior.
% TODO: add illustration of encoder-decoder mechanism as described in SAM meeting.

The \emph{semantic} precision and recall of the preconditions and effects are computed with respect to a set of trajectories that we call the \emph{test trajectories}. 
A reasonable choice for how to create these test trajectories is by running a planner on a set of test problems with the real action model. Alternatively, one may run a random walk with real action model, starting from random start states. 
To compute the defined above metrics, we compute the TP, TN, FP, and FN of every transition in the test trajectories. 

\subsubsection{solving ratio metric}
\subsubsection{syntactic metric}
%% The file kr.bst is a bibliography style file for BibTeX 0.99c





















\section{Measuring Model Effectiveness and Predictiveness}





, one may argwe argue that it is the least important objective. 



and effects designed to capture dis very similar to the real domain model but with different action names or object types.
planning algorithms are often not able to handle large action models with many parameters due to grounding 


e.g., by learning a model that is similar to the real domain model but with different action names or object types, yet not effective in solving problems in it.


a learned domain model may focus on one objective and ignore the other. 
For example, \roni{TODO: Given an example of a domain model that is not similar to the real domain model but 
enables solving problems in it, while one that is very similar and do not}

Therefore, effective domain models may use different representations of states and actions than of the real domain model. 
This is implicitly done by model-free Reinforcement Learning (RL) algorithms, which learn a policy that maps states to actions without explicitly learning a domain model. Such ``domain models'' are not interpretable yet they are sometimes veyr good at solving problems in the same domain~\cite{rlforplanning-paper}. 



and are not useful for planning.








As noted above, we assume that the input to a domain model learning algorithm is a set of trajectories, which are sequences of observations and actions.

All trajectories are collected by observing an agent acting in some domain from observations a real domain model, which we denote as $\realm$.
We assume all the trajectories given to a domain there exists some ground truth domain model $\realm$ and a space of , 




\subsection{Review of classical action model learning evaluation metrics}




\section{Problem Definition}
\omer{TBD , i think it was discussed earlier. Maybe I need to restructure the paper.}

Input: 
- State representation
- Action representation
- Trajectories
- Planner
- Action model learning algorithm


How to compare performance of different algorithms?








For the predicted applicability , 




score to compute the applicability of actions in the learned model with respect to the real model. 

 $\app_M(a,S)$ the set of states in $S$ in which $a$ is applicable according to $M$. Based on this notation, we define the following predicted action applicability metric as follows for some action $a$:

\paragraph{Predicted applicability}
\begin{itemize}
    \item TP$_{\app}(a)=|\app_M(a,S)\cap \app_\realm(a,S)|$
    \item FP$_{\app}(a)=|\app_M(a,S)\setminus \app_\realm(a,S)|$ 
    \item TN$_{\app}(a)=|S\setminus (\app_M(a,S)\cup \app_\realm(a,S))|$
    \item FN$_{\app}(a)=|\app_\realm(a,S)\setminus \app_M(a,S)|$
\end{itemize}
The rationale is ...
\paragraph{Predicted effects}
\begin{itemize}
    \item TP$_{\eff}(s,a)=|(a_M(s)\setminus s)\cap (a_\realm(s)\setminus s)|$
    \item TN$_{\eff}(s,a)=|s \cap a_M(s) \cap a_\realm(s)|$
    \item FP$_{\eff}(s,a)=|(a_M(s)\setminus s)\setminus a_\realm(s)|$
    \item FN$_{\eff}(s,a)=|(a_\realm(s)\setminus s)\setminus a_\realm(s)|$
\end{itemize}
The rationale is ...



Using these functions, we can now describe and use the earlier mentioned evaluation metrics to systematically compare the learned action model \( m \) with the ground truth model \( m^* \).

\subsubsection{semantic metric}
\paragraph{True Positives (TP):}
The set of instances where for every pre-state $s$ and action $a \in A_{m^*}$, if $a$ is applicable in $s$, then at least one learned action $a \in encode(s, a, m^*, m)$ is applicable in $s$.
\paragraph{False Positives (FP):}
The set of instances where there exists a learned action $a \in A_m$ that is applicable in a pre-state $s$, but there is no corresponding action $ a^* \in\ A_{m^*}\ $ $s.t.$ $a^*=decode(s,a',m^*,m) \ $ and $a^*$ is applicable in $s$.
\paragraph{False Negatives (FN):}
The set of instances where for some pre-state $s$ and action $a^* \in A_{m^*}$ $s.t.$ $a^*$ is applicable in $s$, but no learned action $a \in encode(s, a, m^*, m)$ is applicable in $s$.
\paragraph{True Negatives (TN):}
The set of instances where for every pre-state $s$, if a proxy action $a \in A_m$ is not applicable in $s$, then there is no corresponding action $a* \in A_{m^*}$ such that $a$ is applicable in $s$.

\begin{table}[h]
    \centering
    \begin{tabular}{|c|c|c|}
        \hline
        \textbf{Action Applicability} & \textbf{Original Model \( m^* \)} & \textbf{Learned Model \( m \)} \\
        \hline
        True Positive (TP) & Applicable & Applicable \\
        False Positive (FP) & Not Applicable & Applicable \\
        True Negative (TN) & Not Applicable & Not Applicable \\
        False Negative (FN) & Applicable & Not Applicable \\
        \hline
    \end{tabular}
    \caption{Comparison of action applicability in the original and learned models.}
    \label{tab:action-model-comparison}
\end{table}
These metrics allow us to quantify the \emph{precision} and \emph{recall}, where \emph{precision} is $\frac{TP}{TP+FP}$ 
and \emph{recall} is $\frac{TP}{TP+FN}$ This quantification will be useful when choosing the action model learning method based on restrictions and use cases of the learned model. These metrics provide a comprehensive evaluation of the learned model's ability to replicate the original model's behavior.
% TODO: add illustration of encoder-decoder mechanism as described in SAM meeting.

The \emph{semantic} precision and recall of the preconditions and effects are computed with respect to a set of trajectories that we call the \emph{test trajectories}. 
A reasonable choice for how to create these test trajectories is by running a planner on a set of test problems with the real action model. Alternatively, one may run a random walk with real action model, starting from random start states. 
To compute the defined above metrics, we compute the TP, TN, FP, and FN of every transition in the test trajectories. 










 Motivation: representation may be different learning
Our main premise is that the input trajectories are given in a specific domain representation, and the agent for which we aim to generate plans is designed to accept actions specified in that domain representation. 
Still, comparing action model learning algorithms is challenging because different algorithms may output models with varied syntactic structures, representations, or levels of abstraction, making direct comparisons difficult. 

As noted above, computing the metrics defined above requires the learned domain model and the real domain model to be syntactically similar. This is not the case always the case. 
Some action model learning algorithm modify the names of the actions, the ordering of the parameters, the object types, and the predicates in the action model. 
Some algorithms even output action models that are not in the lifted representation, but rather in a grounded representation. 
This makes it challenging to compare the output of different domain learning algorithms. 
Next, we propose a simple and effective solution to this problem, which is based on the idea of using an encoder-decoder mechanism. 




isApplicable(state, groundedAction, model, encode)
possible-grounded-action-list=[]
if encode:
    possible-grounded-action-list = encode(groundedAction)  
else:
     possible-grounded-action-list = [groundedAction]

if any(isApplicable, possible-grounded-action-list) return true
else false


compareApplicability(state, original-grounded-action, OriginalModel, LearnedModel)

    
get\_possible\_ecodings(real\_action, model) 
/// return all actions in the model that my be decoded into real\_action
/// This is a helper function, that may or may not be provided

get\_all\_applicable(state, model)
    applicable\_actions = []
    for every action in model
        if pre(action) in state
            add action to applicable\_actions
    return applicable\_actions
    
        
IsApplicable(real\_state, real\_grounded\_action, model) 
    state = encode(real\_state)
    for grounded\_action in get\_all\_applicable(state, model)  
        if decode(grounded\_action)=real\_grounded\_action 
            return True
    return False

Metric:
    TP,FP,FN,TN=0

    for every (state,action) in data set    
        If IsApplicable(state, grounded\_action, learned\_model)
            If IsApplicable(state, grounded\_action, real\_model)
                TP++
            Else
                FP++
        Else
            If IsApplicable(state, grounded\_action, real\_model)
                FN++
            Else
                TN++
    

    
