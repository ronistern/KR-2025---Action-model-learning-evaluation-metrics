%File: anonymous-submission-latex-2026.tex
\documentclass[letterpaper]{article} % DO NOT CHANGE THIS
\usepackage[submission]{aaai2026}  % DO NOT CHANGE THIS
\usepackage{times}  % DO NOT CHANGE THIS
\usepackage{helvet}  % DO NOT CHANGE THIS
\usepackage{courier}  % DO NOT CHANGE THIS
\usepackage[hyphens]{url}  % DO NOT CHANGE THIS
\usepackage{graphicx} % DO NOT CHANGE THIS
\urlstyle{rm} % DO NOT CHANGE THIS
\def\UrlFont{\rm}  % DO NOT CHANGE THIS
\usepackage{natbib}  % DO NOT CHANGE THIS AND DO NOT ADD ANY OPTIONS TO IT
\usepackage{caption} % DO NOT CHANGE THIS AND DO NOT ADD ANY OPTIONS TO IT
\frenchspacing  % DO NOT CHANGE THIS
\setlength{\pdfpagewidth}{8.5in} % DO NOT CHANGE THIS
\setlength{\pdfpageheight}{11in} % DO NOT CHANGE THIS
%
% These are recommended to typeset algorithms but not required. See the subsubsection on algorithms. Remove them if you don't have algorithms in your paper.
\usepackage{algorithm}
\usepackage{algorithmic}

%
% These are are recommended to typeset listings but not required. See the subsubsection on listing. Remove this block if you don't have listings in your paper.
\usepackage{newfloat}
\usepackage{listings}
\DeclareCaptionStyle{ruled}{labelfont=normalfont,labelsep=colon,strut=off} % DO NOT CHANGE THIS
\lstset{%
	basicstyle={\footnotesize\ttfamily},% footnotesize acceptable for monospace
	numbers=left,numberstyle=\footnotesize,xleftmargin=2em,% show line numbers, remove this entire line if you don't want the numbers.
	aboveskip=0pt,belowskip=0pt,%
	showstringspaces=false,tabsize=2,breaklines=true}
\floatstyle{ruled}
\newfloat{listing}{tb}{lst}{}
\floatname{listing}{Listing}
%
% Keep the \pdfinfo as shown here. There's no need
% for you to add the /Title and /Author tags.
\pdfinfo{
/TemplateVersion (2026.1)
}

% DISALLOWED PACKAGES
% \usepackage{authblk} -- This package is specifically forbidden
% \usepackage{balance} -- This package is specifically forbidden
% \usepackage{color (if used in text)
% \usepackage{CJK} -- This package is specifically forbidden
% \usepackage{float} -- This package is specifically forbidden
% \usepackage{flushend} -- This package is specifically forbidden
% \usepackage{fontenc} -- This package is specifically forbidden
% \usepackage{fullpage} -- This package is specifically forbidden
% \usepackage{geometry} -- This package is specifically forbidden
% \usepackage{grffile} -- This package is specifically forbidden
% \usepackage{hyperref} -- This package is specifically forbidden
% \usepackage{navigator} -- This package is specifically forbidden
% (or any other package that embeds links such as navigator or hyperref)
% \indentfirst} -- This package is specifically forbidden
% \layout} -- This package is specifically forbidden
% \multicol} -- This package is specifically forbidden
% \nameref} -- This package is specifically forbidden
% \usepackage{savetrees} -- This package is specifically forbidden
% \usepackage{setspace} -- This package is specifically forbidden
% \usepackage{stfloats} -- This package is specifically forbidden
% \usepackage{tabu} -- This package is specifically forbidden
% \usepackage{titlesec} -- This package is specifically forbidden
% \usepackage{tocbibind} -- This package is specifically forbidden
% \usepackage{ulem} -- This package is specifically forbidden
% \usepackage{wrapfig} -- This package is specifically forbidden
% DISALLOWED COMMANDS
% \nocopyright -- Your paper will not be published if you use this command
% \addtolength -- This command may not be used
% \balance -- This command may not be used
% \baselinestretch -- Your paper will not be published if you use this command
% \clearpage -- No page breaks of any kind may be used for the final version of your paper
% \columnsep -- This command may not be used
% \newpage -- No page breaks of any kind may be used for the final version of your paper
% \pagebreak -- No page breaks of any kind may be used for the final version of your paperr
% \pagestyle -- This command may not be used
% \tiny -- This is not an acceptable font size.
% \vspace{- -- No negative value may be used in proximity of a caption, figure, table, section, subsection, subsubsection, or reference
% \vskip{- -- No negative value may be used to alter spacing above or below a caption, figure, table, section, subsection, subsubsection, or reference








%############ USER COMMANDS
\usepackage{amsmath}
\usepackage{amsthm}
\usepackage{xspace}
\usepackage{xcolor}
\usepackage{acronym}
\usepackage{multirow}
\usepackage{makecell}
\usepackage{footnote}
\usepackage{pifont}
\usepackage{subcaption}
\usepackage{multicol}

\acrodef{SAM}{Safe Action Model Learning}

\newcommand{\pre}{\ensuremath{\textit{pre}}\xspace}
\newcommand{\eff}{\ensuremath{\textit{eff}}\xspace}
\newcommand{\app}{\ensuremath{\textit{app}}\xspace}

\newcommand{\psynpre}{\ensuremath{P^\pre}\xspace}
\newcommand{\psyneff}{\ensuremath{P^\eff}\xspace}
\newcommand{\psempre}{\ensuremath{P^{sem}_\pre}\xspace}
\newcommand{\psemeff}{\ensuremath{P^{sem}_\eff}\xspace}
\newcommand{\rsynpre}{\ensuremath{R^\pre}\xspace}
\newcommand{\rsyneff}{\ensuremath{R^\eff}\xspace}
\newcommand{\rsempre}{\ensuremath{App}\xspace}
\newcommand{\rsemeff}{\ensuremath{R^{sem}_\eff}\xspace}

\newcommand{\realm}{{\ensuremath{M^*}}\xspace}

\newcommand{\cmark}{\ding{51}} % check mark
\newcommand{\xmark}{\ding{55}} % cross mark
\newcommand{\tuple}[1]{\ensuremath{\left \langle #1 \right \rangle }}

\newcommand{\noop}{\textit{NO-OP}\xspace}
\newcommand{\sam}{\ac{SAM}\xspace}
\newcommand{\masam}{\ac{MA-SAM}\xspace}
\newcommand{\cmasam}{\text{MA-SAM\ensuremath{^+}}\xspace}
% \newcommand{\mfmap}{\ac{MF-MAP}\xspace}
\newcommand{\jat}{\ac{JAT}\xspace}
\newcommand{\blmaa}{\ac{LMA}\xspace}
\newcommand{\blmaas}{LMAs\xspace}
\newcommand{\pbl}{pb-literal\xspace}
\newcommand{\pbls}{pb-literals\xspace}
\newcommand{\learnblmaa}{Learn\blmaa}
\newcommand{\nolam}{NOLAM\xspace}
\newcommand{\offlam}{OffLAM\xspace}
\newcommand{\rosame}{ROSAME\xspace}
\newcommand{\samshort}{SAM\xspace} 
\newcommand{\T}{T}
\newcommand{\Ttrain}{\T_{train}}
\newcommand{\Ttest}{\T_{test}}


\newcommand{\miniparagraph}[1]{\textbf{#1.}}  
\newcommand{\commentout}[1]{ }
\newcommand{\smallcite}[1]{\begingroup\scriptsize\citep{#1}\endgroup}  



\newcommand{\encodea}{\textit{EncodeA}\xspace}
\newcommand{\encodes}{\textit{EncodeS}\xspace}
\newcommand{\decodea}{\textit{DecodeA}\xspace}
\newcommand{\decodes}{\textit{DecodeS}\xspace}
\newcommand{\stest}{\ensuremath{S_{\textit{test}}}\xspace}
\newcommand{\ptest}{\ensuremath{\Pi_{\textit{test}}}\xspace}
% \newcommand{\solved}{\textit{solved}}\xspace}}
\newcommand{\dbtest}{\ensuremath{T_{\textit{test}}}\xspace}



\newif\ifaddcomments
\addcommentstrue % Uncomment this line to remove the user comments


\newcommand{\todo}[1]{\ifaddcomments{\textcolor{red}{[TODO: #1]}}\fi}
\newcommand{\roni}[1]{\ifaddcomments{\textcolor{red}{[Roni: #1]}}\fi}
\newcommand{\argaman}[1]{\ifaddcomments{\textcolor{blue}{[Argaman: #1]}}\fi}
\newcommand{\omer}[1]{\ifaddcomments{\textcolor{purple}{[Omer: #1]}}\fi}
\newcommand{\mauro}[1]{\ifaddcomments{\textcolor{green}{[Mauro: #1]}}\fi}
\newcommand{\yarin}[1]{\ifaddcomments{\textcolor{teal}{[Yarin: #1]}}\fi}
\newcommand{\gregor}[1]{\ifaddcomments{\textcolor{orange}{[Gregor: #1]}}\fi}
\newcommand{\cm}[1]{\ifaddcomments{\textcolor{olive}{[Christian: #1]}}\fi}
\newcommand{\leo}[1]{\ifaddcomments{\textcolor{pink}{[Leonardo: #1]}}\fi}
\newcommand{\brendan}[1]{\ifaddcomments{\textcolor{brown}{[Brendan: {#1}]}}\fi}
\newcommand{\pascalJr}[1]{\ifaddcomments{\textcolor{cyan}{[Pascal L.: {#1}]}}\fi}
\newcommand{\pascalSr}[1]{\ifaddcomments{\textcolor{blue!80!black}{[Pascal B.: {#1}]}}\fi}

% \newcommand{\roni}[1]{ }
% \newcommand{\argaman}[1]{ }
% \newcommand{\omer}[1]{ }
% \newcommand{\mauro}[1]{ }
% \newcommand{\yarin}[1]{ }
% \newcommand{\gregor}[1]{ }
% \newcommand{\cm}[1]{ }
% \newcommand{\leo}[1]{ }

%############






\setcounter{secnumdepth}{0} %May be changed to 1 or 2 if section numbers are desired.

% The file aaai2026.sty is the style file for AAAI Press
% proceedings, working notes, and technical reports.
%

% Title

% Your title must be in mixed case, not sentence case.
% That means all verbs (including short verbs like be, is, using,and go),
% nouns, adverbs, adjectives should be capitalized, including both words in hyphenated terms, while
% articles, conjunctions, and prepositions are lower case unless they
% directly follow a colon or long dash
\title{Evaluating Planning Model Learning Algorithms}
\author{
    %Authors
    % All authors must be in the same font size and format.
    Written by AAAI Press Staff\textsuperscript{\rm 1}\thanks{With help from the AAAI Publications Committee.}\\
    AAAI Style Contributions by Pater Patel Schneider,
    Sunil Issar,\\
    J. Scott Penberthy,
    George Ferguson,
    Hans Guesgen,
    Francisco Cruz\equalcontrib,
    Marc Pujol-Gonzalez\equalcontrib
}
\affiliations{
    %Afiliations
    \textsuperscript{\rm 1}Association for the Advancement of Artificial Intelligence\\
    % If you have multiple authors and multiple affiliations
    % use superscripts in text and roman font to identify them.
    % For example,

    % Sunil Issar\textsuperscript{\rm 2},
    % J. Scott Penberthy\textsuperscript{\rm 3},
    % George Ferguson\textsuperscript{\rm 4},
    % Hans Guesgen\textsuperscript{\rm 5}
    % Note that the comma should be placed after the superscript

    1101 Pennsylvania Ave, NW Suite 300\\
    Washington, DC 20004 USA\\
    % email address must be in roman text type, not monospace or sans serif
    proceedings-questions@aaai.org
%
% See more examples next
}

%Example, Single Author, ->> remove \iffalse,\fi and place them surrounding AAAI title to use it
\iffalse
\title{My Publication Title --- Single Author}
\author {
    Author Name
}
\affiliations{
    Affiliation\\
    Affiliation Line 2\\
    name@example.com
}
\fi

\iffalse
%Example, Multiple Authors, ->> remove \iffalse,\fi and place them surrounding AAAI title to use it
\title{My Publication Title --- Multiple Authors}
\author {
    % Authors
    First Author Name\textsuperscript{\rm 1},
    Second Author Name\textsuperscript{\rm 2},
    Third Author Name\textsuperscript{\rm 1}
}
\affiliations {
    % Affiliations
    \textsuperscript{\rm 1}Affiliation 1\\
    \textsuperscript{\rm 2}Affiliation 2\\
    firstAuthor@affiliation1.com, secondAuthor@affilation2.com, thirdAuthor@affiliation1.com
}
\fi


% REMOVE THIS: bibentry
% This is only needed to show inline citations in the guidelines document. You should not need it and can safely delete it.
\usepackage{bibentry}
% END REMOVE bibentry

\begin{document}

\maketitle

\begin{abstract}
Formulating domain models for model-based planning is a challenging, time consuming, and error-prone task. A number of approaches have been proposed to automatically learn domain models from a given set of observations. A key question is how to compare models learned by different approaches. 
Currently, there are no standard evaluation metrics or benchmarks. 
In this paper, we describe a set of domain model metrics designed to assess different characteristics of a learned domain model. We then present a benchmark suite based on domain models from the International Planning Competition (IPC) and an evaluation process for using it.
Four domain model learning algorithms are evaluated on this benchmark, which highlights the importance of the diverse evaluation metrics we proposed. 
% Then, we propose an encoder-decoder mechanism for comparing two domain models even if they rely on different representations of states and actions. 
% We demonstrate how this encoding mechanism can be leveraged to systematically compare models across different representations using the proposed metrics. 
\end{abstract}

% Uncomment the following to link to your code, datasets, an extended version or similar.
% You must keep this block between (not within) the abstract and the main body of the paper.
% \begin{links}
%     \link{Code}{https://aaai.org/example/code}
%     \link{Datasets}{https://aaai.org/example/datasets}
%     \link{Extended version}{https://aaai.org/example/extended-version}
% \end{links}


\roni{AAAI Instructions: Submissions may consist of up to 7 pages of technical content plus additional pages solely for references and the reproducibility checklist; acknowledgements should be omitted from papers submitted for review.}


\section{Introduction}



% \roni{TODO: What is automated planning. Focus: classical planning. }
Domain-independent planning is a foundational area of research in Artificial Intelligence (AI) that focuses on the automatic generation of plans to achieve specific goals from a given initial state in a given environment. 
\emph{Classical planning}, which is the focus of this work, is the colloquial name for a well-studied type of domain-independent planning in which a single agent is acting in a fully observable, discrete, and deterministic environment. 
% \roni{TODO: Creating an action model is hard. Learning action models algorithms exists.}
Most research on classical planning has focused on developing efficient algorithms for solving planning problems, and assumed the existence of a \emph{domain model} specified in a formal language such as the Planning Domain Definition Language (PDDL)~\citep{mcdermott1998pddl}. 
% \pascalJr{Citation is weird. I think it should be \citep{mcdermott1998pddl}.}RONI: GOOD CATCH!
The domain model in classical planning defines how states are represented, the set of possible actions, and the preconditions and effects of each action. 
However, creating a domain model is a challenging, time-consuming, and error-prone task \citep{DBLP:conf/kcap/McCluskeyVV17}.
This is a bottleneck for the wider dissemination of planning technology in real-world applications. 

% \roni{TODO: Learning action models output action models that use different representations of states and actions.}
To address this issue, a number of algorithms have been proposed to automatically learn domain models from a set of provided observations~\citep{macq,aineto2019learning,jimenez2012review}.\footnote{See http://macq.planning.domains for more details.} 
% \footnote{A comprehensive list of such algorithms is given in https://macq.planning.domains.} 
% is given in \url{https://macq.planning.domains}.} 
This task is often referred to as \emph{domain model learning} or \emph{model acquisition}. %, or \emph{action model learning}. SPACE
% \yarin{ To bring consistency to the terminology used in this area, we can refer to the field as Domain Model Learning (DML), which encompasses what has previously been described as planning model learning, action model learning, or model acquisition.} Roni: let's discuss this in our meeting in the morning. I am not sure  
% gösgens2025learningfromonlyactiontraces,LAMANNA2025104256,xi2024neuro,mordoch2024safe,juba2021safe,lamanna2021online,cresswell2011generalised,ZHUO20101540}.\roni{Everyone is invited to add their action model learning algorithm here.}
% \cm{Why not just point to some of the survey stuff? On the MACQ website (https://macq.planning.domains/), we have the bib entry and pointer to 3 previous survey papers. I'm biased, but I think MACQ should be included :P.} \roni{Good idea, and I definitely think MACQ should be included. This is a mistake on my end, but will be remedied}
% \roni{TODO: Key question: how to evaluate action model learning algorithms?}
Despite a recent resurgence in interest in learning domain models, there is 
no set of agreed-upon evaluation metrics, 
no standard evaluation process for such algorithms, 
and no standard benchmark. This can be seen in Table~\ref{tab:metric-using-comparison}, which lists the metrics used by prior works. 
This paper aims to close this gap, and proposes an evaluation paradigm for domain model learning algorithms that includes a set of metrics, a publicly-available benchmark for evaluation, and a detailed description of how to use it. 




A commonly used, straightforward evaluation process for domain model learning algorithms is based on comparing the \emph{syntactic similarity} of the learned domain model to a \emph{reference domain model}. 
We discuss the limitations of this evaluation method and propose an alternative evaluation process that does not require a reference domain model. 
This evaluation process aims to evaluate the \emph{predictive power} and \emph{problem-solving} ability of the learned domain model. 
Several specific metrics have been defined for this type of evaluation in prior works~\citep{aineto2019learning,juba2021safe,mordoch2024safe,Oswald2024DLLMDomainModeling}, and we discuss their complementary strengths and weaknesses. 
The proposed evaluation process and metrics are the first contribution of this work.

The second contribution of this work is a benchmark suite and evaluation framework based on domain models from the International Planning Competition (IPC). 
This evaluation framework implements the evaluation process described above on this benchmark and outputs the corresponding predictive power and problem solving metrics. 
We implemented this evaluation framework and executed it on several domain model learning algorithms, including \sam \citep{juba2021safe}, Offline Learning of Action Models (OffLAM) \citep{LAMANNA2025104256}, Noisy Offline Learning of Action Models (NOLAM) \citep{Lamanna24}, and ROSAME~\cite{xi2024neuro}. 
This provides reference results and implementations for future research. 
In addition, the results highlight the necessity for the different metrics we proposed, as one model may show impressive results in one metric and poor performance in another. \roni{Is this clear enough?}
The code, dataset, and evaluation process described in this work are intended to be publicly available.\footnote{A link to this is omitted to preserve anonymity. However, the code is available in the supplementary material.} 
Lastly, we briefly discuss how one may implement a more general evaluation framework that bridges over differences between how domain model learning algorithms represent the environment. 



% % \roni{TODO: Examples: grounded vs. lifted, different action names, different object types, paramter ordering, etc.}
% A challenge for using the proposed evaluation process and proposed metrics is that they are designed for evaluating models that use the same representation of states and actions. 
% Domain model learning algorithms, however, may output models that use different representations of states and actions. 
% For example, some algorithms output models in a grounded representation~\citep{stern2017efficient}, while others output models in a lifted representation~\citep{juba2021safe,xi2024neuro,LAMANNA2025104256}. 
% Some represent states based only on the parameters of executed actions~\citep{cresswell2011generalised}, or different languages to support human engineers \citep{mccluskey2010action}. 
% Finally, some require a richer symbolic representation of states~\citep{juba2021safe, Lamanna24}, while others use a visual representation of states~\citep{asai2022classical, xi2024neuro}.
% % \roni{All: add references and refine the above and add more examples of representation gap.}
% We propose an evaluation paradigm that addresses this representation gap challenge. 
% In this paradigm, a domain model learning algorithm under evaluation is obliged to define encoder and decoder functions to bridge the representation gap. 
% We adapt the domain model evaluation metrics defined above to use these bridging functions, 
% and show how to define such functions for several domain model learning algorithms. 





\section{Background}

% In this section, we provide the necessary background on classical planning, approaches to automatically learn domain models, and a discussion on domain models comparison. 
% \subsection{Classical Planning}
% We focus on \emph{classical planning} problems, which is a well-studied type of planning problem in which a single agent is acting in a fully observable, discrete, and deterministic environment. RONI THIS WAS SAID IN THE INTRODUTION
A \emph{classical planning} problem is a tuple $\mathcal{P} = \tuple{F, A, s_0, G}$, where $F$ is a finite set of fluents, $A$ is a finite set of actions, $s_0$ is the initial state, and $G\subseteq F$ is a set of fluents. 
% \roni{For all: I modified the above definition to:(1) not have $S$ as the set of states but have $F$ as the set of fluents, and (2) not have $G$ as a set of states but as a subset of fluents (that must be achieved). If anyone objects please do not change but rather comment here on the change you would like and why.}
A \emph{state} is defined by a set of propositions, representing that the conjunction of fluents in this set are true in this state.
An action $a\in A$ is a tuple $a = \tuple{\mathit{pre}(a), \mathit{eff}(a)}$, where $\mathit{pre}(a)$ is the precondition of $a$ and $\mathit{eff}(a)$ is the effect of $a$. 
The precondition $\mathit{pre}(a)$ specifies the conditions that must hold in a state for the action $a$ to be applicable. The effect $\mathit{eff}(a)$ specifies the changes to the state resulting from applying the action $a$.   
The precondition and effect of an action are defined each as a set of literals, which are either positive or negative propositions. Let $L$ be the set of literals, i.e., 
$L = F \cup \{\neg f \mid f\in F\}$.
% $L=\{\ell \mid \exists f\in F: \ell=f \vee \ell=\neg f\}$. \roni{Not sure about the }
% \pascalJr{Maybe just $L = F \cup \{\neg f \mid f\in F\}$? -- I am not entirely sure if this what you mean. If it is, we could just say that literals are the set of fluents extended by their negations.} roni: I liked the formal version you wrote. Incorporated it in.
% Discussion on lifted domain representation
Planning domains are usually represented in a \emph{lifted representation}, where the actions are defined with respect to a set of object types.
The lifted representation of a planning domain is defined as a tuple $\mathcal{D} = \tuple{O, P, A}$, where $O$ is a set of object types, $P$ is a set of predicates, and $A$ is a set of actions. Actions and predicates are parameterized by objects, and preconditions and effects are defined accordingly.
% \roni{I'm missing a sentence on what is a grounded action}\yarin{
A \emph{grounded action} is obtained by replacing the parameters of an action with concrete objects in the current problem. 
A solution to a classical planning problem is a sequence of grounded actions that transforms the initial state $s_0$ into a goal state $s_g$ where $G\subseteq s_g$. 
% Popular classical planning systems, such FastDownward~\citep{helmert2006fast}, support such a lifted representation. OBVIOUS

% \subsection{Domain Learning Algorithms}
% Action model learning algorithms
Different algorithms have been proposed for learning domains for classical planning~\citep{macq,aineto2019learning,jimenez2012review}. 
The input to these algorithms is a set of \emph{trajectories}. 
A \emph{trajectory} is a sequence of observations and actions. 
An \emph{observation} can be a state or some other information about the state, 
such as a set of predicates that hold in the state or a visual representation of a state. 
Domain model learning algorithms differ in the type of trajectories they learn from, the type of models they can learn, 
% the representation of the learned action model they produce, 
and the learning methodology they apply. 

% Example of prominent domain learning algorithms. 
% \todo{Open request for all: refine and add more algorithms to the following paragraphs as needed. Do not be shy: add your algorithms here.}
%\paragraph{}

% \cm{I'm actually not sure any of this is needed. We're probably pushing over 50 related works in the area (cf. MACQ or similar surveys), and there's no way we'll cover the full space of approaches. A few exemplary ones, and then a pointer to a survey should be fine. Unless we need the background of how some of these work, it's about half-a-page of space we could save. Now the \textit{metrics} used in those papers, is something related.}
% \roni{I agree this needs to be shortened. TODO for me. Would be very good to have a list of the metrics used in each paper. Great idea. TODO by someone? a table or so with metric and list of papers using it?}\yarin{I'm on it}

In this work, we focus on arguably the most common type of domain model learning algorithms for planning, in which the domain model being learned is a classical planning domain, and the input trajectories are also given in a symbolic representation. 
% We refer to this setting as the \emph{classical} domain model learning problem. 
Examples of such learning algorithms include ARMS~\citep{yang2007learning}, which models the learning problem as a MAX-SAT problem; Simultaneous Learning and Filtering (SLAF)~\citep{amir2008learning}, which applies logical inference to filter inconsistent action models; FAMA~\citep{aineto2019learning}, which models the learning problem as a planning problem; and ROSAME~\cite{xi2024neuro}, which trains a deep neural network to output a PDDL domain.
While this area of research is very active~\citep{juba2021safe,mordoch2023safe,xi2024neuro,Lamanna24,LAMANNA2025104256} there is no established benchmark or standardized metrics, as can be seen in Table~\ref{tab:metric-using-comparison}. We will discuss later what these metrics are and how they are computed. 


LOCM~\citep{cresswell2011generalised} and SIFT~\citep{gosgens2024learning} represent a conceptually different type of domain model learning algorithm that rely on \emph{action traces}. Action traces are sequences of actions, without any knowledge of the underlying states. These algorithm learn a state representation that is based on the ``state'' of the objects that can be action parameters. Thus, it is not clear how can one apply them to states that were not observed in the training set without some mapping between an observed state and the internal representation of it according to LOCM/SIFT. Bridging this gap is a topic for future research.  
\roni{Super important: read the above paragraph and comment/edit}

% of consider traces of actions as input

% \roni{Discuss, maybe here, how LOCM cannot be applied. Add discussion on SIFT and LOCM. Below is a draft of thoughts and text. Do not read yet. }
% LOCM~\citep{cresswell2011generalised} and SIFT~\citep{gosgens2024learning} represent a conceptually different type of domain model learning algorithm that consider traces of actions as input. .... 
% \yarin{
% We can add something like this as future work:
% Our evaluation framework currently assumes access to observable states, enabling metrics like action applicability and effect prediction. However, methods like LOCM and SIFT learn action models without explicit state representations, inferring structure solely from action argument sequences. Since these models operate over internal representations without a shared state space, applying our metrics directly is non-trivial. Extending our framework to support such cases by developing a universal encoder/decoder mechanism is an important direction for future work.
% }

% and LOCM~\citep{cresswell2011generalised} and its extensions~\citep{cresswell2013acquiring,gregory2016domain}, which constructs state machines for the objects in the problem. 
% \yarin{SIFT~\citep{gosgens2024learning}, like LOCM, learns planning models from action traces alone. It adds stricter consistency rules to manage errors and infers action preconditions via exhaustive checks over action–predicate patterns, which allows it to be sound and complete, but requires exponential runtime.}
% \yarin{need to add related work on ROSAME (maybe what comment out below)+ to the table}
% \roni{Up to here draft of thoughts and text on LOCM and SIFT.}


% \roni{Here will come Yarin's table on who did what}


% LatPlan~\citep{asai2022classical} and ROSAME-I~\citep{xi2024neuro} are conceptually different since they learn propositional action models from trajectories where the states in the given trajectories are given as images, as opposed to a conjunction of fluents. 
% % \yarin{Rewrote and added part on LatPlan here:}\roni{Great! tnx}
% % LatPlan uses a variational autoencoder with the Gumbel-Softmax trick~\citep{jang2017categorical} to convert image states into discrete propositional symbols, enabling classical planning in latent space.
% LatPlan is a fully unsupervised system that uses a variational autoencoder as a differentiable approximation to convert image states into discrete propositional symbols, enabling classical planning in a learned latent space.
% % ROSAME-I~\citep{xi2024neuro} is a recent algorithm that learns action models from trajectories where the states are given as images.
% % ROSAME-I, in contrast, propose a neural network architecture that learns the action model in a lifted representation in tandem with a neural network that learns to map images to a given symbolic representation. 
% ROSAME-I, in contrast, assumes a predefined set of propositions and action signatures. It simultaneously trains a classifier to identify propositions from images and learns a lifted, first-order action model over the given symbolic vocabulary.
% While both approaches learn from image-based trajectories, LatPlan is fully unsupervised, whereas ROSAME-I incorporates prior symbolic knowledge to learn structured models.

% \renewcommand{\citep}[1]{[\citenum{#1}]} % citeyear

\begin{table}[bt]
\centering
\resizebox{\columnwidth}{!}{
\begin{tabular}{|p{3.7cm}|c|c|c|c|c|}
% \hline
\hline % toprule
\multirow{2}{*}{\textbf{Algorithm}} & \multirow{2}{*}{\textbf{\makecell{Syn.\\Sim.}}} 
& \multicolumn{2}{c|}{\textbf{Predict. Pow.}} 
& \multicolumn{2}{c|}{\textbf{Prob. Solv.}} \\
\cline{3-6}
\textbf{} & \textbf{} & \textbf{App.} & \textbf{Eff.} & \textbf{Solv.} & \textbf{Failed} \\ % \textbf{Syntactic Similarity} \textbf{Predicted applicability} & \textbf{Predicted effects} \textbf{Plan failed ratio}
\hline  % midrule

% \multicolumn{1}{|c|}{} & \multicolumn{1}{c|}{\textbf{Syntactic}} & \multicolumn{2}{c|}{\textbf{Predictive Power}} & \multicolumn{2}{c|}{\textbf{Problem Solving}}\\
% \hline
% \textbf{Alg + Ref.} & \textbf{Similarity} & \textbf{Applicability} & \textbf{Effects} & \textbf{Solving ratio} & \textbf{Failed ratio} \\ % \textbf{Syntactic Similarity} \textbf{Predicted applicability} & \textbf{Predicted effects} \textbf{Plan failed ratio}
% \hline

% ARMS\footnote{\cite{yang2007learning}} & V & X & X & X & X  \\
ARMS \smallcite{yang2007learning} & \cmark & \xmark & \xmark & \xmark & \xmark  \\
\hline
SLAF 
\smallcite{amir2008learning}$^{(*)}$& \xmark & \xmark & \xmark & \xmark & \xmark  \\
\hline
FAMA 
\smallcite{aineto2019learning} & \cmark & \cmark & \cmark & \xmark & \xmark  \\
\hline
LOCM \smallcite{cresswell2013acquiring} & \xmark & (**) & (**) & \xmark & \xmark  \\
\hline
OffLAM \smallcite{LAMANNA2025104256} & \cmark & \xmark & \xmark & \cmark & \cmark  \\
\hline
NOLAM \smallcite{Lamanna24} & \cmark & \xmark & \xmark & \cmark & \cmark  \\
\hline
% Framer \smallcite{lindsay2017framer} & \cmark\footnotemark[\value{footnote}] & \xmark & \xmark & \xmark & \xmark  \\ RONI: THIS ONE LEARNS FROM NL, NOT FROM SYMBOLIC TRAJECTORIES.
% \hline
\sam \smallcite{juba2021safe} & (***) & \xmark & \xmark & \xmark & \xmark  \\
\hline
Cond-SAM \smallcite{mordoch2024safe} & \cmark & \xmark & \xmark & \cmark & \xmark  \\ % Conditional-SAM
\hline % \bottomrule
% \textbf{Stochastic SAM~\citep{juba2022learning}} &  &  &  &  &  &  \\ % none of them ? theoretic only ?
% \hline
% \textbf{LATPLAN~\citep{asai2018classical}} &  &  &  &  &  &  \\
% \hline
ROSAME \smallcite{xi2024neuro} & \cmark & \xmark & \xmark & \xmark & \xmark   \\
\hline
\end{tabular}
}
\caption{Evaluation metrics used in prior works on domain model learning. These metrics are defined later in this paper. %Sections~\ref{sec:problem-setting} and~\ref{sec:metrics} \yarin{blank ref}. 
(*) Only reported runtime results. (**) Measured how many instances were required to ``converged'', i.e., reached where more data is not helpful, and whether the domain is isomorphic to a reference domain model. (***) Measured the number of trajectories needed until the reference domain model is learned.}
\label{tab:metric-using-comparison}
\end{table}

% In these prior works, the evaluated domain was manually checked with a custom hypothesis to evaluate learning efficiency as well.}


% SLAF: Only on blocksworld
% LOCM: Tyre-world, Blocks World, Driverlog and Freecell. Metrics: convergence (i.e., more data does not change), equivalence: isomorphic transition system to the reference model. Adequacy: if a reference domain is equivalent.. adequate = more or less to predictive power 

% \yarin{I think the footnotes are not necessary for the table}\leo{I would also remove the footnotes}\yarin{LATPLAN and ROSAME-I are not classic model learning and not mentioned much in the paper so I didn't added them}\yarin{I did few passes on each paper, if there is any mistake please do comment}\leo{I am not sure to show this table here because we have not yet explained the predictive and solving metrics }\roni{I'm going to push this to the intro, to strengthen our motivation. This will show: see, everyone is doing a different thing, we need to standardize. Oh, and BTW, the metric most people use is really not good. Hopefully this will provide a nice motivation}


% \roni{To all: do not worry about shortening the background section. I will do this later.}

\section{Problem Setting and Syntactic Similarity}
\label{sec:problem-setting}

% \roni{Terminology change: the environment is $E$, the domain reference model is $\realm$} 
%\mauro{reshaped the section. Please check.}

In this work, we consider the following domain model learning setup. 
% \gregor{Here it seems that $\realm$ is not yet a planning model, but a more general ``some environment''. Maybe we should distinguish between the ``environment'' (i.e. the thing that exists in reality) and the planning domain that it can be represented with? This might allow us to talk more concisely about @Mauro's concerns w.r.t.\ the ability to even know this ground truth model.\\
% E.g.\ say: An agent is acting in an environment $E$, 
% for which we assume that it can be represented as a classical planning domain $\realm$. }
% \roni{I like it. Implementing this change below}
An agent is acting in an environment $E$ that we assume can be represented as a classical planning domain. 
The agent's actions are recorded in a set of trajectories, i.e. sequences of observations and actions. 
The observations and actions in the trajectories are given in a specific representation (action names, parameters, objects, types, etc.), which we refer to as the \emph{input representation}. 
A domain model learning algorithm is given this set of trajectories, and is expected to output a domain model for classical planning. 
That is, this domain model can be given as input to a classical planner, and together with an appropriate problem description, the planner can generate plans with it.
% \gregor{Maybe add: These plans then should be applicable in the environment $\realm$ and should lead to the goal set in the problem description.}\roni{I intentionally did not add this, as it is not always the case and we talk later about cases where we need a ``decoder'' to translate the plans generated by the planner and the thing that can be executed by the agent in the environment.}
%\mauro{clarify here the ground truth model, and discuss issues and alternatives.}
%\roni{Note: changing \emph{real domain model} to \emph{environment}, to clarify that it is not a model. Then discuss what if we do have a verified domain model. TODO}
% \gregor{Add: The core question is then: How good is the model learned by the domain model learning algorithm?} DONE
The core question we consider is: \textbf{How good is the model learned by the domain-model learning algorithm?}



% No great place to add this but I do want to "neutralize" a reviewer from being confused about the difference
Learning a domain model is somewhat related to  \emph{model reconciliation}~\citep[inter alia]{ChakrabortiSZK17,SreedharanHMK19} and \emph{model repair}~\citep{bercher2025aSurvey}. 
Model reconciliation is the task of aligning one model with another model to ``explain'' the optimality of generated plans. 
% The usual metric considered there is usually 
The typically adopted metric is 
the number of changes to be made to align the models with the made plan observations, with the goal of minimizing modifications.
Model repair is the task of modifying an initial model according to new types of constraints or observations. Domain model learning can be viewed as an extreme case of model repair where the initial model is empty. 
% \yarin{, but, the objectives differ. Therefore, assessing model repair requires evaluation measures specifically designed to capture its unique objectives and constraints.}\roni{I'm not sure the objective fully differ, so...}
The nature of the repair task is different, however, and thus metrics and evaluation for model repair are expected to be different as well. 
% \roni{I'm not super happy with my last 2 sentences.}  
% \leo{maybe there is some relation between model repair and model learning from an input (possibly flawed) model?} I think I'll stay with these sentences despite their suboptimality :)
% Constraints can come under different forms, such as the request to include/exclude some given plans in the solution space of a model, to ensure that a given plan is optimal for the model, or to constraint the solution space. 

% This relationship also extends to approaches such as model reconciliation~\citep[inter alia]{ChakrabortiSZK17,SreedharanHMK19}, where one model needs to be aligned with another model to ``explain'' the optimality of generated plans. In reconciliation approaches, the usual metric considered is the number of changes to be made to align the models on the made plan observations, with the goal of minimizing modifications.





% Syntactic similarity metrics
\subsection{Syntactic Similarity Metrics}
% \paragraph{Assumptions} 
Many prior works on learning domain models evaluated the learned domain by syntactically comparing the learned domain model to a \emph{reference domain model} (or ground truth model), denoted by $\realm$. This type of evaluation relies on the following assumptions:
\begin{itemize}
    \item \textbf{A1:} A reference domain model 
    is available using the same representation as the given trajectories, and it accurately captures the relevant aspects of the environment. 
    \item \textbf{A2:} The learned domain model uses the same representation as the given trajectories. 
    % \item \textbf{A3:} The learned domain model uses the same representation as the given set of trajectories. 
\end{itemize}
These assumptions are reasonable when evaluating domain model learning algorithms in some controlled environment, e.g., for research purposes. 



Under these assumptions, several metrics have been proposed to quantify the \emph{syntactic similarity} between the learned domain model and the reference domain model~\citep{aineto2019learning,mordoch2023safe,xi2024neuro,Oswald2024DLLMDomainModeling}.
% \roni{Others who have used, please add citations}
These syntactic similarity metrics typically compare the intersection or difference of the predicates in the actions' preconditions and effects between the learned and reference domain models. We define these common metrics below. 
Let $M$ be the evaluated domain model, and let $a$ be an action. We denote by $\pre_M(a)$ the preconditions of action $a$ according to domain $M$.
% Similarly, we denote by $L_a$ the set of all parameter-bound literals associated with action~$a$.[roni: I don't think this is a good fit for here, following CM's comment]

% \cm{"parameter-bound literals" needs to be defined. Also, at this point, it's not clear what the model learner is given. Are we using the same types? Same objects? Same action signatures? What about same fluent names with different number of parameters? I guess all these questions are why we need new metrics beyond the syntactic ones, but some of the assumptions that go into using these (syntactic) metrics would be useful to stipulate.}\roni{Hmm. You're right there's some blunder in the text above. I removed ``parameter-bound literal'' altogether and stayed with simply literals.}

% Let $M$ and $\realm$ be the evaluated and real domains, respectively, and let $a$ be an action. We denote by $\pre_M(a)$ and $\eff_M(a)$ the preconditions and effects, respectively, of action $a$ according to domain $M$.
 \begin{itemize}
    \item True Positives: TP$_\pre(a)=|(\pre_M(a)\cap \pre_\realm(a)|$
    \item False Positives: FP$_\pre(a)=|((\pre_M(a)\setminus \pre_\realm(a)|$
    \item True Negatives: TN$_\pre(a)=|L \setminus(\pre_M(a)\cup \pre_\realm(a))|$
    % , where $L_a$ is the set of all parameter-bound literals for an action $a$.
    \item False Negatives: FN$_\pre(a)=|(\pre_\realm(a)\setminus \pre_M(a))|$
\end{itemize}
The following standard metrics from statistical analysis can then be computed based on these values for each action:
\begin{itemize}
    \item \textbf{Syntactic Precision}: $P_\pre(a)=\frac{TP(a)}{TP(a)+FP(a)}$
    \item \textbf{Syntactic Recall}: $R_\pre(a)\frac{TP(a)}{TP(a)+FN(a)}$
\end{itemize}
Other metrics, such as Accuracy and F1-score, can also be computed based on these values. 
To obtain an overall precision and recall for preconditions of the entire domain model, one can compute the average of the precision and recall values for all actions:
$P_\text{avg}=\frac{1}{|A|}\sum_{a\in A} P(a)$ and $R_\text{avg}=\frac{1}{|A|}\sum_{a\in A} R(a)$, where $P(a)$ and $R(a)$ are the precision and recall of action $a$, respectively. 
The same metrics can be defined for the effects of actions, with the only difference being that the literals in the effects are used instead of those in the preconditions. 
% As discussed above, while the syntactic similarity metrics are easy to compute, they do not accurately reflect functional differences between learned and original models. In addition, these metrics are not well-defined when the learned model and the original model have different action signatures for actions, predicates, or object types. Next, we present our domain model evaluation paradigm, which includes a set of metrics that measure model predictiveness and effectiveness and overcome representation differences between the learned domain model and the real domain model. 
%\mauro{add here the notion of strong/weak equivalence, and the metrics based on graph similarity}
% \roni{TODO: Add an example}
% \gregor{Proposed this ``pathological'' example}
% \roni{Not sure about this. It is just an example, not a "corner case".}
% \roni{We need to split this example. The first part just explains how to compute the precision and recall. The second part shows why this is a bad metric. The second part should be moved to the be under the ``Drawbacks in Syntactic Similarity'' subsection later.}
As an example, consider a delivery scenario.
The reference domain model has predicates $at$, $in$, and $contains$ to describe that a truck or package is at a location, a package is in a truck, and a truck contains a package, respectively.
The action \textsc{unload} has three parameters $\ell, t,$ and $p$ (i.e., location, truck, and package).
In $\realm$, its preconditions are $\textit{at}(\ell, t)$ and $\textit{in}(p,t)$, and its effects are $\neg \textit{in}(p,t), \neg \textit{contains}(t,p),$ and $\textit{at}(p,t)$.
A learned model could conceivably have the same effects, but with the preconditions $\textit{at}(\ell, t)$ and $\textit{contains}(t,p)$.
Then, it would have syntactic recall and precision for the effects of $1$, but for the preconditions it would be $\frac 1 2$. 
A finer-grained variant of the syntactic similarity metric, proposed by \cite{chrpa2023comparing}, is to assess the \emph{edit distance} of the learned domain model from the reference domain model. Low distance values indicate models that are syntactically close to each other, and if two models are syntactically identical (i.e., edit distance of zero), then they are said to be \textit{strongly equivalent}.





%Similarly, the work done on model reconciliation \citep{} looks into approaches to suggest minimal changes to a model to reconcile it with a reference one.


%\roni{Add: discuss the limitation of the syntactic similarity metrics and the reliance on a reference model.}
% \subsection{Drawbacks in Syntactic Similarity}
The syntactic similarity metrics have several limitations. 
First, they require the existence of a domain reference model $\realm$ (Assumption A1). Such a model is rarely available when applying automated planning technology in real-world applications, where a PDDL specification of the environment is not given. 
This was also the setup in the International Competition on Knowledge Engineering for Planning and Scheduling (ICKEPS) \citep{DBLP:journals/aim/ChrpaMVV17}. 
Second, comparing to a reference model implies there is a single best model for the environment. 
What constitutes a \emph{best model} for an environment is not well-defined, and assessing the quality of a model is very challenging \citep{DBLP:conf/kcap/McCluskeyVV17}. 
In fact, one may say that any domain model that is consistent with the observations is equally likely to be the ``real one'', as the learning algorithm is not given any incentive to prefer one over the other. 
% \roni{@Pascal and Gregor: is this sentence capturing what you wrote about in the comment below?}
Lastly, there may be multiple domain models for a given environment that are identical for all practical purposes, yet different syntactically. 
Thus, it is not clear whether the syntactic similarity of the learned model to a reference model is a good indicator of how \emph{useful} a learned domain model is. 
% This limitation has been observed in prior works~\citep{aineto2019learning,juba2021safe,mordoch2024safe}.
Next, we discuss what constitutes a useful domain model and how to evaluate it without the need for a reference domain model.

% \roni{Tried to compile above the comments from Mauro and Gregor (which are commented out below). Please check if it is ok.}

% First, it relies on the existence of a reference domain model, which may not be available in practice. 
% In fact, this is usually the case in new applications of automated planning to real-world problems. In such cases, there may be a number of alternative models, but not a specific reference one. This is the setting, for instance, of the International Competition on Knowledge Engineering for Planning and Scheduling (ICKEPS) \citep{DBLP:journals/aim/ChrpaMVV17}.
% \gregor{One could distinguish two cases here: (1) the practical application case, i.e. where we want to use domain learning in a real environment/industrial setting. Or (2) where we only want to evaluate the methods w.r.t.\ their usability. In the latter case, we could argue that we have ``exhaustively'' studied the setting we want to evaluate the model learner on and could thus have a planning model that models the situation -- but still in this case, the issue of multiple equivalent models remains.}


% \gregor{Maybe add a new second reason here to prepare the argumentation for the next one:\\
% Second, there might be multiple models that are fully identical for all practical purposes.
% That is, we will be able to find exactly the same set of plans using all of these models.
% A supposed ground truth model would then be an arbitrary pick out of these equivalent models.
% In evaluation would reward a learner if it can reproduce the one arbitrarily chosen ground truth model over any other equivalent model.
% Considering the above transport example, it is equivalent to use $in(p,t)$ or $contains(t,p)$ as a precondition -- but using syntactic similarities forces the learner to have a bias for one of the two option, without being able to infer which bias is correct out of the input trajectories.
% }\roni{I think this is captured in the paragraph below (probably someone editted it - maybe Gregor or Mauro? anyhow it turned out nicely)}

% Second, the reference model is assumed to be of the best possible quality. This is in itself complicated by the fact that it is very challenging to assess the quality of a model \citep{DBLP:conf/kcap/McCluskeyVV17}. Besides this, the use of a reference model biases the assessment towards a single specific encoding -- while many others of similar quality could potentially exist.
% Third, it is not clear that the syntactic similarity of the learned model to a reference model is a good indicator of how \emph{useful} a learned domain model is. This limitation has been observed in prior works~\citep{aineto2019learning,juba2021safe,mordoch2024safe}.
% Next, we discuss what constitutes a useful domain model and how to evaluate it without the need for a reference domain model.




\section{Metrics for Evaluating Domain Models}
\label{sec:metrics}
% \roni{TODO: Add assupmtions on what we know (objects, types, etc.)}DONE
% \gregor{In the previous section, we strongly argue that there might not be a ``best'' reference model of the environment available, which we can base our comparison on. But in this section, we still use the reference model $\realm$ to compare with -- we just do it in a more indirect fashion. Should we start this section with a discussion of what we can still assume in terms of access? And where the parallels to classical machine learning are?\\
% From my point-of-view: (1) Having a true model of the actual environment to compare against is problematic in a realistic setting (why would we try to learn it otherwise?). (2) We can assume to have some samples, i.e., plans available from that environment, maybe also counter example plans. (3) There are two types of evaluations that we can conduct: the ICKEPS style, where we have an actual real-world problem that needs modelling, resp.\ whose model we need to learn (here we don't have a model!) and on the other hand our test-bed evaluation where the sole purpose is to determine whether the learning algorithm is good.\\
% In the second case, we create an artificial environment and specify it using a PDDL domain. And then run our experiments in that domain. For these evals, it is totally ok to assume that we have a model to compare against. But: we should still be representation agnostic, which syntactic measures are not.\\
% Let's take the analogy to ML here: they typically want to analyze how well we can learn from a specific real world setting (and thus have no model). While new ML-techniques can (and are?) also tested on artificial test-bed setting and datasets, where the ground truth, i.e., generation of all the data is known and can be used for evaluation.}
% \roni{I like this distinction. Tried to incorporate it in the text above, by making the assumptions explicit, and then as the paper progresses we remove them (section 4 removes A1, section 5 removes A2). Let me know if I got it right.}



% \gregor{One general problem: the syntactic metrics all work solely on the level of the domain representation. But the metrics we propose here all require some type of actual states -- i.e.\ they are dependent on at least the set of objects that exists. Or even more they are dependent on the full problem instances. I assume we then have a selection of ``test'' problem instances as in standard machine learning?}\roni{Yes, different metrics require differet "test sets". I'll write about this below, let me know if this works}
We consider two main dimensions when evaluating the ``usefulness'' of a learned domain model: 
%its \emph{predictive power} and its \emph{ability to enable solving problems}.
% optimize the learning process and to evaluate the learned model:
%
%In this setup, we consider the following desired objectives of a learned domain model. 
\begin{itemize}
    % \item \textbf{Syntactic similarity.} Aim to learn a domain model that is  as similar as possible syntactically to the reference one. 
    \item \textbf{Predictive power.} Aim to learn a domain model that allows predicting the applicability of actions in the environment and the outcomes of applying them. 
    \item \textbf{Problem-solving ability.} Aim to learn a domain model that enables the generation of applicable plans in the environment within given resource bounds. %enables solving problems in the real domain model.
    % \gregor{Question here is: which solutions? (1) A plan that solves the learned problem? Definitely not, as this is not useful in the real environment. (2) A plan that will be executable in the real environment!}    \roni{Good catch. I rephrased to emphasize it's (2)}
\end{itemize}
%\mauro{reshape discussion below -- align problems-solving ability to operationality notion.}

These dimensions may not be aligned with the syntactic similarity metrics, even if there exists a reference model that enables computing them.  
A domain model may accurately predict when actions are applicable and what the effects of applying them will be, while still being very different from a given reference model. 
For example, if a domain maintains logical invariants on its states, such as mutex (mutual exclusion) conditions, that the reference model does not explicitly define. The learned model may include these invariants as additional preconditions that only rule out the action in environment states that are unreachable in actual trajectories.
% since it encodes mutex conditions that the reference model did not bother to define. \roni{Is this clear?}\yarin{Yes but, how is this instead: A domain model can be highly \emph{predictive} of an agent's behavior in the reference domain, even if it differs significantly from the reference model. This can happen when the learned model encodes mutex (mutual exclusion) conditions that the reference model does not explicitly define.
% }Roni: pluged this in above. 
% \gregor{maybe: ``does only define implicitly''?}\roni{Editted. Is it clearer?} \gregor{I've also put this issue already in the example above. Is this more illustrative?}\roni{Yes, thanks!}
Similarly, a learned domain model may be very \emph{similar syntactically} to a reference domain model but not effective in solving problems from the corresponding real-world environment. 
For example, the learned domain may miss a single crucial effect or include an extra precondition that cannot be met in a key state, which prevents it from solving many problems correctly. 
% adding redundant preconditions or missing a crucial effect \citep{DBLP:conf/kcap/VallatiC19}. \gregor{or missing a crucial precondition: e.g.\ can only open a door if I have key.} [Roni: added it]
In contrast, a domain model may be very different from a reference domain model, e.g., adding many extraneous preconditions to some actions, yet very \emph{effective} in solving problems in the application domain since these extraneous preconditions are often true in the real world. 


The two dimensions described here -- predictive power and problem-solving ability -- are also not necessarily aligned with each other. A domain model may be very \emph{predictive} of behavior of the agent in the reference domain model, yet too complex to be used for solving problems by any existing planning engine.
For example, consider a learned domain model that has many copies of the same action in the reference domain model, each with different parameters and preconditions, in order to capture different aspects of that action's behavior. 
While this may be useful for predicting the applicability of actions in the environment, 
it may hinder the ability of a planning engine to find plans for problems in the application domain with this model, due to its complexity (e.g., large branching factor). 
Therefore, different metrics are required to evaluate these two dimensions. We describe such metrics below. 

Importantly, computing all the metrics described in this section do not require a reference domain model. 
Thus, they can be used to compare two models directly,  providing metrics to assess which one is better according to different aspects of the environment. 
Nevertheless, we still assume the learned model uses the input representation (Assumption A2).  
In addition, computing these metrics requires that it is possible to attempt to perform actions in the environment $E$ and observe if the action is applicable a well as the resulting state (if it is). % (if the action is indeed applicable in the environment). 
% Note that all the metrics described in this section do not require a reference model --- they are computed with respect to the environment. 



% The first dimension is important as it means one can use the learned domain model to validate the correctness of a plans, predict expected outcomes of the agents' actions, and identify faults during plan execution. 
% The second dimension is important as it means one can use the learned domain model to solve planning problems. \roni{Above paragraph: maybe redundant, not sure?}


% \paragraph{Assumptions}
% Computing these metrics still In this section, we still assume the learned model uses the input representation (Assumption A2). 
% While we do not require a reference domain model, we do assume that it is possible to attempt to perform actions in the environment and observe the resulting state (if the action is indeed applicable in the environment). 


% \roni{Maybe the above paragraphs are too lengthy and the point is already clear. Not sure.}

% \roni{I think this can be merged, talking immediately on the case where there is no reference model for these metrics. Not sure about this. Thoughts?}
% We begin by assuming that the learned domain model and the reference domain model both use the same representation of states and actions, %i.e., they are \emph{syntactically similar} 
% in the sense that they have the same action names, object types, and parameter ordering. This is indeed generally the case for approaches that learn models from trajectories and plan traces, that come with a pre-defined vocabulary. 
% In Section~\ref{sec:bridging-gap}, we discuss how to extend the metrics to handle cases where the learned domain model and the reference domain model use different representations. 



\subsection{Predictive Power Metrics}
\label{sec:predictiveness-metrics}
% While the syntactic similarity metrics are easy to compute, they do not accurately reflect the \emph{predictive power} of the learned model. 
%The \emph{predictive power} metrics described next are designed to address this issue. 
The predictive power metrics, also referred to as \emph{semantic} domain model metrics~\citep{aineto2019learning,mordoch2024safe,le2024learning}, measure how well a learned model is able to predict the applicability of actions and their effects in the environment. 

% \cm{Up until this point, I thought it was headed towards the predictive power of explaining the trajectory data. Given that it's about action applicability, this needs to be clarified much earlier on (e.g., intro). Also means that we don't have a notion of how well the learned model predicts the trajectories?}
% \roni{Hmm. I assumed all models are consistent with the given trajectories, so to evaluate it we must have other states and the ability to evaluate with the environment. TODO: clarify this.}
% \cm{I think just stating that assumption, and maybe moving the citations in the above text to the end of the sentence, would do.}
% \roni{I'm not sure. One could conceivably choose an algorithm that is not consistent. I need to think about this more.}


We define two types of predictive power metrics: \emph{action applicability} metrics and \emph{predicted effects} metrics. 
The former measures the ability of the learned model to predict whether an action is applicable in a given state, while the latter measures the ability of the learned model to predict the effects of an applicable action in a given state.
Unlike the syntactic similarity metrics, which are computed based on the action model itself, the predictive power metrics require a dataset of states that we denote by $\stest$. 
This dataset is intended to represent the distribution of states of interest in the domain. 
Thus, using \stest may actually be beneficial over using all the possible states, as planning domains can have many ``unnatural'' states~\cite{grundke2024frepr} which may bias the evaluation. 
E.g., in Blockworld we can theoretically define that two blocks can be on top of each other, i.e., $\operatorname{on}(a, b) \wedge \operatorname{on}(b, a)$, but such a state does not make sense. Thus, if the learned model is inaccurate in such states, it may not be important. 
% Even if a 
% A good reference pointing this out is \cite{grundke2024frepr}.
% $S$ contains all these ``idiotic" states that might bias the evaluation, whereas \stest does not bias it in this way.
% (Though, of course comes with it own caveats.)
Obtaining a representative $\stest$ is a challenge. One way to do so is by observing an agent acting in the environment, e.g., operated by an expert. 

% If a reference domain model exists, one may create such a dataset by running a planner on a set of test problems using the reference domain model. Alternatively, one may create a dataset of states by



% % \gregor{Or to actually observe states that appear in the actual environment during execution.} ADDED ABOVE.
% \cm{We still need the original domain in order to compute the predictive power. Some of the criticisms lofted at the existing metrics may need to be scaled back because of this -- there's no way to know if the predicted actions that are applicable are the right ones, without having a reference model.}
% \roni{We don't need a reference domain but we do need a way to try actions in the environment (Even if it is a black box. Added text above to clarify this.}

% % \roni{TODO: Clarify this does not consider complexity of the domain (which will be addressed in the next metric)}
% \gregor{Issue: what happens if we either cannot get the states themselves, i.e., no symbolic description or if the state description of the actual model differs from the one of the learned model? (say: predicates are named differently?)\\
% Can we be more radical here? Maybe it is sufficient to have plans and non-plans for the real model? Then we could check whether these plans are executable in the learned model and whether the non-plans are not. One issue: this mixes applicability and effects.
% }
% \cm{Ya, I think many of the definitions start to fall apart if we don't have several assumptions down -- same objects, same types, same fluent/action symbols with their signatures, etc.}
% \roni{Added clear text above to state our assumptions. I want to talk about different representations in the next section with the encoder-decoder part.}

% \pascalJr{I believe that it would be interesting to point out here that using \stest can actually be beneficial over using the whole set $S$.
% A planning domain can define many unnatural states. 
% E.g., in Blockworld we can theoretically define that two block can be on top of each other, i.e., $\operatorname{on}(a, b) \wedge \operatorname{on}(b, a)$.
% But, this just doesn't make sense.
% A good reference pointing this out is \cite{grundke2024frepr}.
% $S$ contains all these ``idiotic" states that might bias the evaluation, whereas \stest does not bias it in this way.
% (Though, of course comes with it own caveats.)
% }\roni{I like this angle. Incorporating it in. }

\paragraph{Predicted applicability}
For a domain model $M$ and action $a$, we denote by $\app_M(a,\stest)$ and $\app(a,\stest)$ the set of states in \stest 
% $S$ \pascalJr{\stest?} RONI: thanks!
in which $a$ is applicable according to $M$ and $E$, respectively. 
% \gregor{This is easy to compute for STRIPS/SAS+, but hard if you have things like disjunctive preconditions. It should be \#P-hard}\roni{Hmm. I edited above to clarify the \stest is not the set of al states, only the set of test state we're using. }
Using this notation, we define the following predicted action applicability metrics as follows for some action $a$:
\begin{itemize}
    \item TP$_{\app}(a)=|\app_M(a,\stest)\cap \app(a,\stest)|$
    \item FP$_{\app}(a)=|\app_M(a,\stest)\setminus \app(a,\stest)|$ 
    \item TN$_{\app}(a)=|\stest\setminus (\app_M(a,\stest)\cup \app(a,\stest))|$
    \item FN$_{\app}(a)=|\app(a,\stest)\setminus \app_M(a,\stest)|$
\end{itemize}
% \leo{Should we keep TN? (since is not mentioned in precision/recall), similarly for predicted effects}
% \gregor{Hm. That is strange. Should we also define precision and recall for the negative class? Does that make sense? This would be how good the approach is to determine non-applicable actions. I.e.:
% \begin{align}
%     P_{\app}^-(a)= & \frac{TN_{\app}(a)}{TN_{\app}(a)+FN_{\app}(a)}\\
%     R_{\app}^-(a)= & \frac{TN_{\app}(a)}{TN_{\app}(a)+FP_{\app}(a)}    
% \end{align} 
% }
% \leo{I am not sure since it's binary classification. I think precision/recall for the negative class is dual w.r.t. the positive class ones}
% \roni{Yes, Leonardo is right. But I still want to keep TN for completeness.}
% }

In words, TP$_{\app}(a)$ is the number of states in \stest where $a$ is applicable according to both the learned model and the real model, FP$_{\app}(a)$ is the number of states in $\stest$ where $a$ is applicable according to the learned model but not the real model, TN$_{\app}(a)$ is the number of states in $\stest$ where $a$ is not applicable according to both models, and FN$_{\app}(a)$ is the number of states in $\stest$ 
% \pascalJr{\stest?} TNX!
where $a$ is applicable according to the real model but not the learned model. 
From these metrics, one can compute the precision and recall 
for action applicability
of every action $a$ in the learned model as 
$P_{\app}(a)= \frac{TP_{\app}(a)}{TP_{\app}(a)+FP_{\app}(a)}$ 
and $R_{\app}(a)= \frac{TP_{\app}(a)}{TP_{\app}(a)+FN_{\app}(a)}$, 
respectively. 
% \argaman{Notice that when $TP_{\app}(a)=FP_{\app}(a)=0$ the precision value is set to one by default as there are no actions that are applicable in states where they should not be.}\roni{Rephrasing this to be more definitive}
When $TP_{\app}(a)=FP_{\app}(a)=0$ it means the domain model never allowed $a$ to be applied. We define $P_{\app}(a)$ and $R_{\app}(a)$ to be one and zero, respectively, in such cases. %, reflecting that the model never mistakenly . Such a case corresponds to an action model that never allows $a$ to be applied, and thus The rationale for this design choice is that an action model that never allows $a$ to be applied never allthen it never  in a state where it is inapplicable in the environment, it also never allows $a$ in a state where it is. 






% follows,
% \begin{itemize}
%     \item $P_{\app}(a)= & \frac{TP_{\app}(a)}{TP_{\app}(a)+FP_{\app}(a)}$
%     \item $R_{\app}(a)= & \frac{TP_{\app}(a)}{TP_{\app}(a)+FN_{\app}(a)}$
% \end{itemize}
% \begin{align}
%     P_{\app}(a)= & \frac{TP_{\app}(a)}{TP_{\app}(a)+FP_{\app}(a)}\\
%     R_{\app}(a)= & \frac{TP_{\app}(a)}{TP_{\app}(a)+FN_{\app}(a)}    
% \end{align}  FOR SPACE



\paragraph{Predicted effects}
For domain $M$, action $a$, and state $s$, we denote by $a_M(s)$ and $a(s)$ the state resulting from applying $a$ in $s$ according to $M$ and $E$, respectively. 
Based on this, we define the following predicted action effect metrics for every state $s\in\stest$ and action $a$, as follows: 
\begin{itemize}
    \item TP$_{\eff}(s,a)=|(a_M(s)\setminus s)\cap (a(s)\setminus s)|$
    \item FP$_{\eff}(s,a)=|(a_M(s)\setminus s)\setminus a(s)|$ 
    \item TN$_{\eff}(s,a)=|s \cap a_M(s) \cap a(s)|$
    \item FN$_{\eff}(s,a)=|(a_M(s)\cap s)\setminus a(s)|$ 
\end{itemize}

% \gregor{there is a start too many (this is always $\emptyset$) I assume it should be $(a_M(s)\cap s)\setminus (a_\realm(s) \cap s)$ -- the facts that are not changed by $a_M$, but not counting the ones changes by $M_\realm$}\roni{Good catch! I fixed it in a slightly different way, let me know if you agree or not.}\cm{The TN and FN aren't looking correct...but after spending some minutes with it, I think it is ;). Should state that a "negative" here indicates a literal that remains unchanged.}\gregor{Yep, but it does -- $a_M(s)$ is the next state. If you take the intersection with the previous one, you get exactly all literals that have not changed.}
%     \leo{should we denote $a_M(s)$ as `$s'$ obtained after executing $a$ in $s$`? I'm not sure currently is formally neat since we used $a$ to denote actions in the previous paragraph.}
%     \roni{I think it's Ok and I've seen planning paper use action also as a function (e.g,. $a_i(a_{i-1},\cdots a_1(s_0)/cdots)$ to denote applying a sequence of actions. But, if anyone feels strongly about this we can change to $apply(a,s)$ or something like that. But I think the current form is clean and well-defined.}
For the purpose of the above computation, a state includes all the literals true in it, i.e., both positive propositions and negative ones. 
Importantly, we consider above only state-action pairs $s$ and $a$ where $a$ is applicable in $s$ in the real environment and the learned domain. 
Otherwise, the outcome of applying $a$ in $s$ is not defined. 
% Aggergating
To obtain the predicted effects metrics per action, we simply average over all the states in $\stest$. Similar to the predicted applicability metric, when $TP_{\eff}(a)=FP_{\eff}(a)=0$, we define $P_{\eff}(a)$ and $R_{\eff}(a)$ to be one and zero, respectively.
% As discussed above, one can then aggregate over all actions to compute either the average or the cumulative precision and recall values. 
% or the cumulativeby either averaging over actions
% Then, the precision and recall of the predicted effects for each action can be computed as shown above for predicted applicability. 
% \leo{The following sentence may be a repetition of what we wrote in the previous paragraph?}
% Alternatively, we compute the \emph{cumulative precision and recall} by summing each metric separately. 
% \argaman{Notice, similar to the predicted applicability metric, when $TP_{\eff}(a)=FP_{\eff}(a)=0$, we define $P_{\eff}(a)$ and $R_{\eff}(a)$ to be one and zero, respectively. This reflects there were no states in which the action was applicable, resulting in it not changing any of the states in which it was 'applied' on.}
% \leo{I am not sure, for example when an action is never applicable then is not considered by the predicted effects metric, i.e. there is no TP and FP}\roni{I think we converged yesterday to Argaman's text above?}
% \roni{Ensure this makes sense} \gregor{Checked.}TNX!
% \roni{Aggregate only over action applicable according to both 
% Rationale: if a model specifies an action does not applicable thne its effects are not defined.
% TODO: Add a discussion about this}
% \leo{maybe it may be useful to mention an action where an effect is learned as a precondition, the predicted effect is correct when the action is applicable in the learned model}\roni{Added below text for this}
% \roni{TODO: Move discussion on aggregating till after the predicted effects}
% and the overall action applicability, precision, and recall
% by averaging over all actions.  

% \roni{After some thought, I don't think the discussion about the difference between predicted effects and syntactic effects is good. The syntactic does not measure w.r.t a test set of state while the predicted effects does, and does not need a reference domain.}
% The difference between the syntactic similarity of effects and the predicted effects metrics is subtle. 
% It manifests when considering an action $a$ applied in a state $s$ that contains some literal $\ell$ is observed in the state before an action $a$, 
% it is not an 
% \gregor{posisitve i.e.\ adding?} effect according to the learned model, but it is an effect according to the real model. 
% This will count as a false positive in the syntactic similarity metrics yet not count as a false positive in the predicted effects metrics. 



% One can aggregate the above metrics over all states and actions, \gregor{This might be computationally a problem -- so we sample?!}\roni{Hmm. I am not sure. Is it worse than low-order polynomial in $S$ and number of actions?}
% \gregor{No, but $|S|$ can be exponential in the number of facts/predicates that we have.}
% \roni{You're right. I had at some stage some text on separating $S$ from the set of all states, having the set of states be implicitly defined by the set of fluents. I modified the PDDL definition above, and will add that we need a "test set" of states somewhere here.}
% \leo{I agree, currently in the implemented evaluation the `test set` is the set of states in a *test* set of trajs obtained interleaving optimal planning and random actions. A point is that some actions are executable in a few states, which are unlikely to be sampled randomly; moreover random sampling can generate invalid states}\roni{Added some text above on this.}
% compute the overall values, and compute aggregated precision and recall values accordingly.  



% \gregor{One more argument for these predictive power metrics overall: They allow us to differentiate between ``cautious'' or ``safe'' learning (e.g. \cite{juba2021safe}) and optimistic approaches (e.g. \cite{Bachor24learning}; that is don't add a precondition until proven it must exist). Safe model learning will always have $FP_{app}(a) = 0$ at the cost of high $FN_{app}(a)$, while optimistic learning will have $FN_{app}(a) = 0$, but high $FP_{app}(a)$.\\
% Which of the two is better depends on your concrete application --- in critical domains (air traffic, medicine) you want safe model, but in cases where you want to only get some plan that might work and you can always backtrack if it actually does not (transport? travelling?) you are ok with this.}
% \roni{This is a nice discussion. Can you add it in?}\roni{Wish we had an optimistic model learning algorithm implemented so we can run it on the benchmark.}


% \roni{Add a concrete example of how to compute these metrics}
% \yarin{I added example in section\_4\_example.tex, I don't think it's that good}\roni{Ok, removing this. }


\paragraph{Aggregated Precision and Recall}
% \leo{Since there is no space, and we already mention the average aggregation in the syntactic metrics paragraph, I think we could mention also the `simple` aggregation there and remove this paragraph. Then after describing predicted applicability and predicted effects we can say the aggregation can be done in the same way}

% \leo{Moreover in the syntactic metrics subsection we refer to the \emph{equal-action} as \emph{average}, maybe we can keep that term, and replace \emph{simple} with \emph{cumulative} or \emph{overall}? Since in the first case we aggregate by averaging between actions, in the second case we sum up TP/FP/FN over all actions}
% \roni{Great idea, implementing it now.}
The above precision and recall metrics are computed per action. It is often beneficial to aggregate over all actions and obtain precision and recall metrics for predicted applicability or predicted effects over the entire learned domain. 
One way to do so is to \emph{average} the precision and recall obtained for each action. 
Alternatively, one may \emph{sum} the 
TP, FP, TN, and FN separately, and then compute the precision and recall. 
That is, compute precision and recall based on $TP_\app=\sum_a TP_\app(a)$, 
$FP_\app=\sum_a FP_\app(a)$,
$TN_\app=\sum_a TN_\app(a)$, and
$FN_\app=\sum_a FN_\app(a)$. 
We refer to the result of the first aggregation method as the \emph{average precision and recall} of the chosen metric (predicted applicability or predicted effects). 
We refer to the result of the second aggregation method as the \emph{cumulative precision and recall} of the chosen metric.  
Both are reasonable options. 
In our evaluation framework described below, we used the average aggregation method. 
%\roni{@Leonardo, which one we used?}
% \leo{For syntactic precision the `simple` one, this is because e.g. if there are 2 actions with 1 and 10 precs/effs resp., then the precision averaged over the two actions weights 1 mistake in the first action differently from 1 mistake in the second one, I am not sure we want this. For the predictive power metrics we aggregated using the `equal-action` method, this is because the number of samples can be very unbalanced (e.g. 1 vs 100). I do not think there is a `right` one, so I stored both metrics during the experiments and we can choose}
% \roni{In the syntactic metrics this is not an issue because we measure based on the model, not the states in the test set. }



% \gregor{I wanted to add one more point that I got from Pascal Bercher:\\
% If we view domain model learning as an algorithmic problem, we are supposed to only judge the output of the learner based on the inferences that can be drawn explicit from the given trajectories.
% If we do anything else we implicitly evaluate the bias of the learners towards the type domain models we ``usually'' have in the input domains.
% }
% \roni{I do not fully understand the last comment.}\roni{Got it now}
 
\subsection{Problem-Solving Metrics}
%\mauro{add here disclaimer}\roni{Added the disclaimer below. Let me know what you think}
Neither the syntactic similarity metrics nor the predictive power metrics are sufficient for evaluating the \emph{operationality} \citep{DBLP:conf/kcap/McCluskeyVV17} of the learned domain model, i.e., its ability to solve problems. It is well-known that even small syntactical changes in domain models can result in significant performance gaps \citep{DBLP:conf/kcap/VallatiC19,vallati2021importance}. 
%effectiveness of using a learned domain model to actually solve problems in the application domain. 
We propose two metrics that are designed to measure a model's ability to solve problems: \emph{solving ratio} and \emph{false plan ratio}. 
Both metrics are defined with respect to a set of problems \ptest in the environment $E$. 
The solving ratio metric is defined as the fraction of problems in \ptest that can be solved with the learned model by a given planning within a fixed limit on the available computational resources --- CPU runtime and memory.
We say that a problem is \emph{solved} if a plan is found within the allowed computational resources, and that plan is valid according to the environment. 
The false-plan-ratio metric is defined as the fraction of problems in \ptest that are solved by the learned model but are not valid according to the environment. This reflects the reliability of the learned model for producing plans. 
Additional problem-solving metrics can also be considered that quantify the runtime and solution quality of returned solutions. 

% Limitations
The straightforward nature of the above metrics is not without limitations. The ability to solve a problem with a given domain depends on external factors such as the set of test problems, the planner used, the runtime and memory budget allowed for it to run, and the computer and OS that executed the planner. 
% \gregor{Theorem provers and proof assistants, e.g., Lean, sometimes use ``heartbearts'' (\url{https://florisvandoorn.com/carleson/docs/Lean/Util/Heartbeats.html}) instead of time. They measure elementary orations and bound their number. This ensures reproducibility across different machines. For us this could e.g.\ be number of expanded states. Or the number of times an effect is applied -- this would also include computation effort for heuristics.}
% \roni{I am not sure I fully understand. I know that in search people often count nodes expanded instead (or in addition to) time because expanded nodes is less affected by what the OS is doing. But doing something similar here is not trivial, I think?}
Ideally, the above metrics would be run on a diverse set of test problems, planners, resource limits, and computing machines. In practice, this might be difficult to implement, but one is advised to at least run all evaluated domains on the same setup and provide an appropriate disclaimer to the concluded result. 

% \commentout{I THINK THIS DISCUSSION THREAD IS DONE?
% \roni{TODO: Add metric on the runtime of solving the problem.}
% \roni{TODO: Discussion: what about a metric on how many real plans can be validated by the learned model.}\mauro{Yes! validation capability of the learned model with regards to plans generated with the reference model!}
% \roni{TODO: Maybe: add some unsolvability detection metrics?}
% \cm{In our work on aligning (which is very closely related, it seems!), we computed both (1) if the plans found using P and M validate on M* and (2) if the plans found using P and M* validate on M. You need assumptions on the action names+parameters, but then can test (via VAL) both ways.}
% \gregor{Agree! I have been using this in teaching as well: generate plans for both $P$ and $\realm$.}

% \gregor{Proposal:\\
% Let $M$ and $\realm$ be the evaluated and reference domain models, respectively, and let $a$ be an action.
% Let $\mathcal P(M,k)$ be the set of all plans of $M$ that have at most $k$ many actions (or cost $k$) and that do not visit the same state $s$ more than once during their execution.
% We then define three metrics:
% \begin{itemize}
%     \item $TP_{plan}(k) = |\mathcal P(M,k) \cap \mathcal P(\realm,k)|$
%     \item $FP_{plan}(k) = |\mathcal P(M,k) \setminus \mathcal P(\realm,k)|$
%     \item $FN_{plan}(k) = |\mathcal P(\realm,k) \setminus \mathcal P(M,k)|$
% \end{itemize}
% Lastly we could define $FN_{plan}(k) = |A|^k - TP_{plan}(k) - FP_{plan}(k) - FN_{plan}(k)$.
% %
% We only consider plans up to length $k$, as computing the set of all plans that a model admits might be computationally infeasible.
% To determine $\mathcal P(M,k)$, we could use a top-k planner~\cite{katz2018novel,speck2020symbolic,von2022loopless}.
% Alternatively one can generate plans for one model and validate it in the other using, e.g., VAL.\\
% The main advantage of this evaluation metric is that it is (mostly) agnostic w.r.t.\ the representation of the problem chosen by the learner.
% E.g.\ if the learned model uses different predicates or an entirely different formalism (numeric, images, ...) to represent states.
% The only technical requirement is that we are able to ``translate'' the initial state and goal states of the problems in $\realm$ to the problems in $M$.\\
% This however does not solve the issue of potentially different representations of the actions, which we will discuss in the next section.
% }
% \leo{I have currently implemented the alternative: generate plans with a `reference` model and validate them in the learned model. I liked Gregor proposal to define also this metric in terms of TP/FP/FN, I would slightly rephrase it as: 
% \begin{itemize}
%     \item $TP_{plan} = |\Pi(s, G) \cap \Pi^*(s, G)|$
%     \item $FP_{plan} = |\Pi(s, G) \setminus \Pi^*(s, G)|$
%     \item $FN_{plan} = |\Pi^*(s, G) \setminus \Pi(s, G)|$
% \end{itemize}
%  where $\Pi$ and $\Pi^*$ are the set of solution plans in $M$ and $M^*$ for going from state $s$ to a goal state in $G$}
% \roni{I see a distinction here between measuring our model's ability to understand the environment, which falls into the category of its predictive power, and the ability of using our model to solve problems. 
% In this subsection, which talks about solving problems with a planner, I don't think a top-k plans is a good fit. Maybe we want to add to the predictive power subsection an another metric that measures how well the model enables understanding valid plans in the environment and detecting and flaws in them}



%All the metrics above are defined based on the assumption that the ground truth model is available. 
%However, in many cases, the ground truth model is not available. 
%Moreover, the ground truth model may be a suboptimal representation for the actual environment. 

% \cm{I think the above is misleading. We need to have states (fluents matching the learned model), action applicability (for predictive power), etc, etc. We do need a reference model! What's changed is that we aren't using syntactic comparisons anymore, but the M* model is certainly still there.}
% \gregor{I think you could relax this a bit more. Iirc the point here is that you don't need to have a symbolic model of $\realm$ to evaluate. It would be enough to ``try'' the actions in the actual physical world. For action applicability this would be enough.
% %
% Well in both cases, you still need to be able to correlate an initial state in the real world with a symbolically represented initial state in the learned model.
% For example, if you observe only action sequences (think LOCM), you have no means to correlate the initial state that your learner thought the trajectories have with the actual one. You are missing a kind-of anchoring or ``grounding'' in the actual model $\realm$.
% This is what Roni's group proposes in the Encoder-Decoder Mechanism section.}
% \roni{I now explicitly say in the beginning of this section that we assume the same representation (Assumption A2). But, in the next section we relax it. }
% For the predictive power metrics, we can use the same dataset of states $S$ to compute the predicted applicability and predicted effects metrics for each evaluated model. Similarly, the problem-solving metrics can be computed for each model without a reference model, if one can execute the plans generated by the learned model in the real environment, then we can evaluate the solving ratio and false plans ratio metrics. 


%If two models are generated from scratch and compared



% particularly since certain predicates or negative conditions in one model might implicitly imply others, complicating direct comparisons and similarly there exists domains (rovers for example) that rely on add-delete of effects mechanism where it adds and deletes the same literal $l$ in the same transition to simulate a continuous domain but it cannot be represented in the trajectory used as an input for the learning algorithms. Furthermore, another limitation arises when learned domain actions or predicates do not share identical signatures with those of the original domain, making direct syntactic comparisons even more challenging.


\section{Benchmarks and an Evaluation Process}

% \roni{Terminology discussion: so far we did not use the term ``operator''. Here we do. I think you mean here lifted action? if so, let's use the same terminology all over. I dislike operator as I can never remember if operator is the grounded version of the lifted one.}\leo{sure I replaced it with `lifted action`}

% \todo{Leonardo is in charge of this section.} Argaman also 

\begin{table}[ht]
\centering
\resizebox{\columnwidth}{!}{
\addtolength{\tabcolsep}{-0.2em}
    \centering
    \begin{tabular}{|l|c|c|c|c|c|c|c|c|c|c|}
    % \toprule
    \hline
    Domain & $|A|$ & $|P|$ & Types & Const. & \multicolumn{2}{c|}{$A$ arity} & \multicolumn{2}{c|}{$P$ arity} & Neg. & Inj. \\ 
     &  &  & & & $min$ & $max$ & $min$ & $max$ & pre.  & ass. \\ \hline
    % \midrule
barman & 12 & 15 & 9 & no & 2 & 6 & 1 & 2 & no & yes \\ %
blocksworld & 4 & 5 & 1 & no & 1 & 2 & 0 & 2 & no & yes \\ %
childsnack & 6 & 13 & 6 & yes & 2 & 4 & 1 & 2 & no & no \\ %
depots & 5 & 6 & 9 & no & 3 & 4 & 1 & 2 & no & yes \\ %
elevators & 6 & 8 & 5 & no & 3 & 5 & 2 & 2 & no & yes \\ %
ferry & 3 & 5 & 2 & no & 2 & 2 & 0 & 2 & no & yes \\ % 
floortile & 7 & 10 & 3 & no & 3 & 4 & 1 & 2 & no & yes \\ %
goldminer & 7 & 12 & 1 & no & 1 & 2 & 0 & 2 & yes & no \\ %
grippers & 3 & 4 & 4 & no & 3 & 4 & 2 & 3 & no & yes \\ %
matchingbw & 10 & 10 & 2 & no & 2 & 3 & 1 & 2 & yes & yes \\ %
miconic & 4 & 6 & 2 & no & 2 & 2 & 1 & 2 & no & yes \\ %
nomystery & 3 & 6 & 5 & no & 3 & 6 & 2 & 3 & no & yes \\ %
npuzzle & 1 & 3 & 2 & no & 3 & 3 & 1 & 2 & no & yes \\ %
parking & 4 & 5 & 2 & no & 3 & 3 & 1 & 2 & no & yes \\ %
rovers & 9 & 25 & 7 & no & 2 & 6 & 1 & 3 & no & no \\ %
satellite & 5 & 8 & 4 & no & 2 & 4 & 1 & 2 & yes & yes \\ %
sokoban & 2 & 4 & 3 & no & 3 & 5 & 1 & 3 & no & yes \\ %
spanner & 3 & 6 & 5 & no & 3 & 4 & 1 & 2 & no & yes \\ %
tpp & 4 & 7 & 7 & no & 3 & 7 & 2 & 3 & no & no \\ %
transport & 3 & 5 & 6 & no & 3 & 5 & 2 & 2 & no & yes \\ %
\hline

    % \hline
    \end{tabular}
    }
    \caption{Details about the benchmark domains that are relevant to domain model learning algorithms. The columns from left to right correspond to the name of the domain; the number of lifted actions, lifted predicates, and types; if there are constants; the min. and max. arity of the actions; the min. and max. arity of the predicates; if a literal can be a delete effects without being a preconditions; and if the 
    injective binding assumption holds~\citep{juba2021safe}, i.e., 
    whether a grounded action may map the same object to multiple parameters.}
    \label{tab:domains}
\end{table}


%\subsection{Benchmark}
We describe a process for evaluating domain model learning algorithms 
and computing the metrics defined above. % and propose a corresponding set of domain model learning problems as a benchmark. 
% by experimenting with state-of-the-art approaches on a newly generated benchmark for offline learning of classical planning domains.
%
As a standard benchmark, we propose to use the set of $20$ classical planning domains adopted in all previous IPC learning tracks \citep{fern2011first, vallati20152014, taitler20242023}. 
These domains serve as both the environment and a reference domain model (as required by syntactic similarity metrics). 
We chose these domains for simplicity, as they are well known in the planning community, which developed problem generators necessary for benchmarking. 
The number of lifted actions, predicates, and object types in these domains varies in 
$[1, 12]$, $[3,25]$, and $[1, 9]$, respectively.
Further details about the benchmark domains are provided in Table~\ref{tab:domains}. 

% \footnote{\label{foot:supplementary} Further details about the benchmarks are provided in the supplementary material.\roni{If space permits, let move these details back in, as the benchmark is part of our contributions}}
%
\paragraph{Learning the domain from the training data} 
The first step in any domain model learning evaluation process is to obtain training data $\Ttrain$, i.e., trajectories recording interactions with the environment. 
In our evaluation process, we generate these training trajectories by simulating the behavior of an agent in the environment, mixing goal-oriented and exploration-oriented behavior. 
This is done as follows. 
First, we generate feasible problems using existing generators \citep{seipp-et-al-zenodo2022}. 
Then, we solve the generated problem using an automated planner with the reference domain model (from the IPC). 
The agent then begins to follow the generated plan, but interleaves a random action with some probability $p_{rnd}$.  
After executing a random action, a new plan is computed by the planner from the resulting state (again using the reference domain model). If the random action leads to an unsolvable state, we backtrack and perform a different action. 

To generate diverse goal-oriented behavior, we use a greedy planner configuration for some problems and an optimal configuration for others. 
Let $p_{opt}$ be the ratio of problems in which the optimal configuration is used. 
To emphasize the importance of using both optimal and suboptimal planners to generate trajectories, consider the following example. 
In the \textsc{barman} domain, an agent can either \textsc{clean} a previously used shot and then \textsc{fill} it (which requires $2$ actions), or just \textsc{refill} a used shot (which requires only $1$ action); while both alternatives are possible in an heuristic plan, the \textsc{refill} action must be executed in an optimal one. 
Thus, generating both optimal and suboptimal plans increases the likelihood of ``sampling'' more diverse set of actions. 
Finally, to produce heterogeneous trajectories, we also generated trajectories from problems with different numbers of objects of each type.\footnote{An exception to this in our benchmark is the \textsc{npuzzle} domain, where there is a single object type, and linearly increasing it leads to problems that are too difficult to be solved.} 


% We also randomized the trajectory lengths in $[5, 30]$, and the number of objects in $[3, 107]$\footnotemark[1].
%


In our experiments, we generated $10$ trajectories using the above process with $p_{rnd}=0.2$ and $p_{opt}=0.3$. 
The obtained set $\Ttrain$ of trajectories includes every lifted action at least once, and every trajectory includes a number of objects in $[3, 107]$ and states in $[5, 45]$. %\footnotemark[3].
% In our experiments, the optimal configuration was used in 30\% of the problems. 
For the greedy configuration, we used the FastDownward planner \citep{helmert2006fast} with lazy greedy best-first search, the
FF heuristic \citep{hoffmann2001ff}, and the context-enhanced additive heuristic \citep{helmert2008unifying}. For the optimal configuration, we used $A^*$ with the LM cut heuristic. % \roni{@Leonardo: what was the optimal configuration?} \leo{Astar with landmark optimal cut heuristic}


% To increase the chance of including not strictly necessary lifted actions in the set of trajectories of every domain, we \emph{optimally} solved $30\%$ of the problems.}
% \roni{Was this done only for this domain or for all domains?}
% \leo{for all domains, I tried to state it more explicitly}



%Note that a random action is executed only if it does not make the problem unsolvable.

% For each domain, we produced a \emph{training} set $\Ttrain$ of $10$ trajectories from a set of $10$ small-medium sized problems. 
% To obtain every trajectory in $\T_{train}$, we first randomly generated a feasible problem using existing generators \citep{seipp-et-al-zenodo2022}. 
% Then, we ran FastDownward \citep{helmert2006fast} to produce a heuristic solution plan using the IPC domain model, and generated the trajectory by interleaving plan and random actions (with $20\%$ probability) starting from the initial state of the problem. 
% \leo{After executing a random action, a new solution plan is computed from the resulting state. Note that a random action is executed only if it does not make the problem unsolvable.} 
% \roni{Not clear: do you mean that after having a plan we execute it and with 20\% do a random action instead of the planned action? if so, do we then replan from the resulting state}\leo{yes, I tried to clarify it}
% For planning, we adopted a lazy greedy best-first search with the
% FF heuristic \citep{hoffmann2001ff} and the context-enhanced additive heuristic \citep{helmert2008unifying}.
%
% It is worth noting that problem generators available in the literature can be biased in terms of initial states and goals. For example, in the \textsc{ferry} domain, the ferry is always empty in the initial state; in the \textsc{floortile} domain the goal of every planning problems is to paint all even tiles white and all odd ones black, leaving an empty extra row to place the robot at the end. 
% %
% To mitigate such biases, we create random trajectories from the original ones as follows: Given a trajectory $T=\tuple{s_0,a_0,s_1,a_1,s2,...,a_{n-1},s_n}$ we create a new trajectory $T'$ by shuffling and excluding some of the transitions in $T$. Notice, the initial and the final states in $T'$ are also randomized. 
% \argaman{This might not be what we were aiming for...}
% % \emph{subtrajectory} from the original generated one, resulting in a subtrajectory where the initial and final states are also randomized. 
% %
% However, some operators are unlikely to appear in a randomly sampled trajectories. 
% For example, in the \textsc{goldminer} domain, the operator \textsc{pick-gold} is always executed at the end of a solution plan. Hence, we jointly sample the initial and final states of each random trajectory such that the initial or final state are the same as the original trajectory, with a probability $0.33$.   
%

%

% \paragraph{Running the evaluated learning algorithm}
Next, the evaluated domain model learning algorithms run with the generated training set $\Ttrain$. Then, we compute the syntactic similarity metrics, comparing the learned domain with the reference domain from the IPC. 
While these metrics have their inherent limitations as described above, they are common in the literature and thus we still report them in our evaluation process, in addition to the other metrics (i.e., predictive power and problem-solving). 

%



\paragraph{Generating test states and problems}
The next step in our evaluation process is to generate a set of test states ($\stest$) and a set of test problems ($\ptest$). The former is needed for the predictive power metrics and the latter for the problem solving metrics. 
% se setes are needed to compute the predictive power and the problem solving metrics.  (i.e., predicted applicability and predicted effects), 
To generate the set of test states, we create a set of trajectories denoted $\Ttest$ and collect all the states in these trajectories. 
The process of generating $\Ttest$ is similar to that of $\Ttrain$. 
We generate problems in the environment and solve a ratio $p_{opt}=0.3$ of them optimally with the reference domain model, using the same planning configurations adopted for generating $\Ttrain$. 
% \roni{Why did we choose optimally here and not like we did for generating $\Ttrain$?} \leo{to better balance the samples for actions that are not necessarily executed in heuristic plans. The similarity with the previous barman example is to obtain a number of samples of \textsc{refill} closer to other actions}\roni{I didn't fully understand (sorry)}
Then, we create trajectories by simulating an agent that alternates between performing actions in the plan and random actions with $p_{rnd}=0.2$. 
In our experiments, we generated $100$ such trajectories. 
% \roni{Not clear: does this mean after every random action you replanned to update the plan?}


% The agent then begins to follow the generated plan, but interleaves a random action with some probability $p_{rnd}$.  
% After executing a random action, a new plan is computed by the planner from the resulting state (again using the reference domain model). If the random action leads to an unsolvable state, we backtrack and perform a different action. 



% Problems are generated by 
% These trajectories were created in a 
% To generate $\Ttest$, we generated a set of 
% As in the process of generating the training trajectories, we 

% we aimed to evaluate the predictive power 
% To create these set of trajectories, we solved optimally 100 problems in the environment using the reference domain model. Then we created trajectories by alternating between performing actions in the optimal plan and random actions. \roni{Not clear: does this mean after every random action you replanned to update the plan?}

% we 
% generated a \emph{test} set $\stest$ composed by states in a \emph{test} set $\T_{test}$ of $100$ trajectories. The trajectories in $\Ttest$ were produced by optimally solving $100$ small-sized problems, and interleaving plan and random actions as for $\Ttrain$. Note $\Ttest$ differ from $\Ttrain$.\roni{I am confused. Is this what was written above?}
% \leo{I rephrased/restructured this, now should be clearer}
%
To compute the problem-solving metrics, we generate a test set \ptest in the evaluated domain and run a planner with the learned domain model. 
While not mandatory, we suggest limiting \ptest to only include problems that are solved by the reference domain in the allotted computational resources. 
In our experiments, we generated $10$ problems for every domain and only considered problems that FastDownward could solve with the reference domain model within a time limit of $60$ seconds and $16$ GB of RAM. %the same resource limitations.\roni{What were our time/memory limitations?}\leo{60 secs time limit, then my PC has 16GB of RAM (which were not fully used), I can measure MB ram usage if necessary}

% We limited \ptest to only include problems that even 

% Since the \emph{solving} and \emph{false plan} ratios are evaluated using a limited amount of resources (e.g.\ a CPU time limit set to $60$ seconds), we only considered problems that FastDownward could solve with the IPC domain model within the same resource limitations.

% of $10$ problems for every domain. Since the \emph{solving} and \emph{false plan} ratios are evaluated using a limited amount of resources (e.g.\ a CPU time limit set to $60$ seconds), we only considered problems that FastDownward could solve with the IPC domain model within the same resource limitations.


%\subsection{Evaluation Paradigm}
%


\subsection{Experimental Results}
% 
% \begin{figure}[t]
% % \label{fix:exp-results}
%   \centering

%   \begin{subfigure}[b]{0.6\columnwidth}
%     \includegraphics[width=\textwidth]{figures/10_traces_mini/syn_precision.png}
%     \caption{Syntactic precision}
%   \end{subfigure}
%   \\
%   % \begin{subfigure}[b]{0.24\textwidth}
%   %   \includegraphics[width=\textwidth]{figures/10_traces_mini/syn_recall.png}
%   %   \caption{Syntactic recall}
%   % \end{subfigure}
%   \begin{subfigure}[b]{0.6\columnwidth}
%     \includegraphics[width=\textwidth]{figures/10_traces_mini/solving.png}
%     \caption{Problem solving ratio}
%   \end{subfigure}
%   \\
%   \begin{subfigure}[b]{0.6\columnwidth}
%     \includegraphics[width=\textwidth]{figures/10_traces_mini/app_recall.png}
%     \caption{Applicability recall}
%   \end{subfigure}




%   \caption{Evaluation metrics when learning from a training set $\Ttrain$ with $10$ traces for every domain.}
%   \label{fig:exp-mini}
% \end{figure} 


% OPTION B

\begin{figure*}[ht]
% \label{fix:exp-results}
  \centering

  \begin{subfigure}[b]{0.3\textwidth}
    \includegraphics[width=\textwidth]{figures/10_traces_mini/syn_precision.png}
    \caption{Preconditions syntactic precision}
    \label{fig:syn-precision}
  \end{subfigure}
  \hfill
  % \begin{subfigure}[b]{0.24\textwidth}
  %   \includegraphics[width=\textwidth]{figures/10_traces_mini/syn_recall.png}
  %   \caption{Syntactic recall}
  % \end{subfigure}
  \begin{subfigure}[b]{0.3\textwidth}
    \includegraphics[width=\textwidth]{figures/10_traces_mini/solving.png}
    \caption{Problem solving ratio}
    \label{fig:solving-ratio}
  \end{subfigure}
  \hfill
  \begin{subfigure}[b]{0.3\textwidth}
    \includegraphics[width=\textwidth]{figures/10_traces_mini/app_recall.png}
    \caption{Applicability recall}
  \end{subfigure}




  \caption{Evaluation metrics for the models learned by \samshort{}, \offlam{} and \nolam{} from a training set $\Ttrain$ of $10$ traces for every domain.}
  \label{fig:exp-mini}
\end{figure*} 
\input{figures/exp_10traces_bars}

\begin{figure*}[ht]
% \label{fix:exp-results}
  \centering

  \begin{subfigure}[b]{0.45\textwidth}
    \includegraphics[width=\textwidth]{figures/10_traces/lineplots/syn_precision_line.png}
    \caption{Syntactic precision}
    \label{fig:syn-precision}
  \end{subfigure}
  % \hfill
  \begin{subfigure}[b]{0.45\textwidth}
    \includegraphics[width=\textwidth]{figures/10_traces/lineplots/syn_recall_line.png}
    \caption{Syntactic recall}
  \end{subfigure}

  \vspace{1em}

  \begin{subfigure}[b]{0.45\textwidth}
    \includegraphics[width=\textwidth]{figures/10_traces/lineplots/app_precision_line.png}
    \caption{Applicability precision}
  \end{subfigure}
  % \hfill
  \begin{subfigure}[b]{0.45\textwidth}
    \includegraphics[width=\textwidth]{figures/10_traces/lineplots/app_recall_line.png}
    \caption{Applicability recall}
  \end{subfigure}

  \vspace{1em}

  \begin{subfigure}[b]{0.45\textwidth}
    \includegraphics[width=\textwidth]{figures/10_traces/lineplots/predeffs_precision_line.png}
    \caption{Predicted effects precision}
  \end{subfigure}
  % \hfill
  \begin{subfigure}[b]{0.45\textwidth}
    \includegraphics[width=\textwidth]{figures/10_traces/lineplots/predeffs_recall_line.png}
    \caption{Predicted effects recall}
  \end{subfigure}

  % \vspace{1em}

  % \begin{subfigure}[b]{0.45\textwidth}
  %   \includegraphics[width=\textwidth]{figures/10_traces/lineplots/solving_line.png}
  %   \caption{Problem solving ratio}
  %   \label{fig:solving-ratio}
  % \end{subfigure}
  % % \hfill
  % \begin{subfigure}[b]{0.45\textwidth}
  %   \includegraphics[width=\textwidth]{figures/10_traces/lineplots/false_plans_line.png}
  %   \caption{False plans ratio}
  %   \label{fig:false-positive-plans}
  % \end{subfigure}

  % \caption{Overall caption for the 6 images.}
  % \caption{Evaluation metrics when learning from a training set $\Ttrain$ with $10$ traces for every domain.}
\caption{Evaluation metric values for the models learned by \sam, \offlam, \nolam, and \rosame in every benchmark domain. The training set $\Ttrain$ includes $10$ trajectories for every domain. \roni{Not sure about these figures. Maybe the previous ones were better? } \leo{I re-added previous ones (i.e. barplots), after adding ROSAME results, the bar plots seem to me a bit overloaded}}
  % \label{fig:exp}
\end{figure*} 

\begin{table}[h]
\resizebox{\columnwidth}{!}{
\addtolength{\tabcolsep}{-0.4em}
\begin{tabular}{l|cccc|cccc|}
% \toprule
\hline
 & \multicolumn{4}{c|}{Solv. \% $\uparrow$} & \multicolumn{4}{c|}{False \% $\downarrow$} \\
Domain & \offlam & \samshort & \rosame & \nolam & \offlam & \samshort & \rosame & \nolam 
\\
% \midrule
\hline
barman & $\mathbf{1.00}$ & $0.80$ & $0.00$ & $0.80$ & $\mathbf{0.00}$ & $\mathbf{0.00}$ & $\mathbf{0.00}$ & $\mathbf{0.00}$ \\
blocksworld & $\mathbf{1.00}$ & $\mathbf{1.00}$ & $0.00$ & $\mathbf{1.00}$ & $\mathbf{0.00}$ & $\mathbf{0.00}$ & $1.00$ & $\mathbf{0.00}$ \\
childsnack & $0.00$ & $\mathbf{1.00}$ & $0.00$ & $0.00$ & $1.00$ & $\mathbf{0.00}$ & $1.00$ & $1.00$ \\
depots & $\mathbf{1.00}$ & $\mathbf{1.00}$ & $\mathbf{1.00}$ & $\mathbf{1.00}$ & $\mathbf{0.00}$ & $\mathbf{0.00}$ & $\mathbf{0.00}$ & $\mathbf{0.00}$ \\
elevators & $\mathbf{1.00}$ & $0.10$ & $0.10$ & $0.10$ & $\mathbf{0.00}$ & $\mathbf{0.00}$ & $\mathbf{0.00}$ & $\mathbf{0.00}$ \\
ferry & $\mathbf{1.00}$ & $\mathbf{1.00}$ & $0.00$ & $\mathbf{1.00}$ & $\mathbf{0.00}$ & $\mathbf{0.00}$ & $1.00$ & $\mathbf{0.00}$ \\
floortile & $\mathbf{1.00}$ & $0.90$ & $0.00$ & $0.90$ & $\mathbf{0.00}$ & $\mathbf{0.00}$ & $\mathbf{0.00}$ & $\mathbf{0.00}$ \\
goldminer & $0.00$ & $\mathbf{1.00}$ & $0.00$ & $\mathbf{1.00}$ & $1.00$ & $\mathbf{0.00}$ & $\mathbf{0.00}$ & $\mathbf{0.00}$ \\
grippers & $\mathbf{1.00}$ & $\mathbf{1.00}$ & $\mathbf{1.00}$ & $\mathbf{1.00}$ & $\mathbf{0.00}$ & $\mathbf{0.00}$ & $\mathbf{0.00}$ & $\mathbf{0.00}$ \\
matchingbw & $\mathbf{1.00}$ & $\mathbf{1.00}$ & $0.00$ & $\mathbf{1.00}$ & $\mathbf{0.00}$ & $\mathbf{0.00}$ & $\mathbf{0.00}$ & $\mathbf{0.00}$ \\
miconic & $\mathbf{1.00}$ & $\mathbf{1.00}$ & $\mathbf{1.00}$ & $\mathbf{1.00}$ & $\mathbf{0.00}$ & $\mathbf{0.00}$ & $\mathbf{0.00}$ & $\mathbf{0.00}$ \\
nomystery & $\mathbf{1.00}$ & $0.00$ & $\mathbf{1.00}$ & $0.00$ & $\mathbf{0.00}$ & $\mathbf{0.00}$ & $\mathbf{0.00}$ & $\mathbf{0.00}$ \\
npuzzle & $\mathbf{1.00}$ & $\mathbf{1.00}$ & $\mathbf{1.00}$ & $\mathbf{1.00}$ & $\mathbf{0.00}$ & $\mathbf{0.00}$ & $\mathbf{0.00}$ & $\mathbf{0.00}$ \\
parking & $\mathbf{1.00}$ & $0.90$ & $0.00$ & $0.90$ & $\mathbf{0.00}$ & $\mathbf{0.00}$ & $\mathbf{0.00}$ & $\mathbf{0.00}$ \\
rovers & $\mathbf{1.00}$ & $0.20$ & $0.00$ & $0.30$ & $\mathbf{0.00}$ & $\mathbf{0.00}$ & $\mathbf{0.00}$ & $\mathbf{0.00}$ \\
satellite & $\mathbf{1.00}$ & $\mathbf{1.00}$ & $\mathbf{1.00}$ & $\mathbf{1.00}$ & $\mathbf{0.00}$ & $\mathbf{0.00}$ & $\mathbf{0.00}$ & $\mathbf{0.00}$ \\
sokoban & $\mathbf{1.00}$ & $\mathbf{1.00}$ & $0.90$ & $\mathbf{1.00}$ & $\mathbf{0.00}$ & $\mathbf{0.00}$ & $\mathbf{0.00}$ & $\mathbf{0.00}$ \\
spanner & $\mathbf{1.00}$ & $\mathbf{1.00}$ & $\mathbf{1.00}$ & $\mathbf{1.00}$ & $\mathbf{0.00}$ & $\mathbf{0.00}$ & $\mathbf{0.00}$ & $\mathbf{0.00}$ \\
tpp & $\mathbf{1.00}$ & $0.00$ & $0.00$ & $0.00$ & $\mathbf{0.00}$ & $\mathbf{0.00}$ & $\mathbf{0.00}$ & $\mathbf{0.00}$ \\
transport & $\mathbf{1.00}$ & $\mathbf{1.00}$ & $\mathbf{1.00}$ & $\mathbf{1.00}$ & $\mathbf{0.00}$ & $\mathbf{0.00}$ & $\mathbf{0.00}$ & $\mathbf{0.00}$ \\
% \bottomrule
\hline
\end{tabular}
}
\caption{Problem solving (Solv.\%) and false plan ratio (False \%) achieved by planning with the domains learned by \offlam, \sam, \rosame, and \nolam; $\uparrow$ (resp. $\downarrow$) indicates the higher (resp. lower) the better, the best metric value for every domain is highlighted in bold.}
\label{tab:solving_ratio_false_plans_ratio}
\end{table}


% \roni{Writing standard: we show here 3 concrete algorithms, not 3 approaches.}

We demonstrate our evaluation process by applying it to four state-of-the-art domain model learning algorithms, namely \samshort~\citep{juba2021safe}, \offlam~\citep{LAMANNA2025104256}, \nolam~\citep{Lamanna24}, and \rosame~\citep{xi2024neuro}. The results of our evaluation are described below. 
% Due to space constraints, only a subset of the results are shown in Figure~\ref{fig:exp-mini}. 
All the experiments were run on a CPU Apple M1 Pro with $16$ GB of RAM.
\leo{need to rerun ROSAME using same resources. TODO: Leonardo}
%We evaluated the learning algorithms on standard benchmark domains using the proposed metrics. 





\miniparagraph{Syntactic similarity} 
% We report syntactic precision and recall using the \emph{simple} aggregation across the preconditions and effects of all actions. \roni{It was written here that we use the cumulative aggregation and not the average here. But, this is a syntactic similarity metric, you can only aggregate over actions. Commenting this out}
\samshort and \nolam exhibited relatively low syntactic precision ($<0.8$) across all domains, as they assume the presence of negative preconditions and include them in the learned domains. 
In contrast, \offlam and \rosame assume that domains do not include negative preconditions, which is a correct assumption for our benchmarks, leading to higher precision values.
%
Nonetheless, \rosame precision drops (i.e., $\le 0.7$) in $5$ out of~$20$ domains due to unnecessary positive preconditions in the learned domains, i.e., positive literals misclassified as preconditions.
%
The effects precision equals $1$ for all algorithms except \rosame, which achieved a precision of $98.85\%$ for positive effects and $96.90\%$ for negative ones, averaged over all benchmark domains. 


%
% In terms of syntactic recall, \samshort's low values stemmed from its inability to learn some actions entirely, i.e., the actions have all the preconditions and no effects, while \offlam and \nolam struggled to learn specific predicates due to either domain inconsistencies (e.g., in \textsc{rovers}) or the presence of constants not bound to action parameters (e.g., in \textsc{childsnack}).
% \roni{All this discusses preconditions. What about the syntactic similarity of effects?}
% \leo{
% In terms of effects, 
For the syntactic recall, \offlam and \nolam obtained effects recall of $1$ in all domains except \textsc{rovers} (because of inconsistent effects), \textsc{goldminer} (due to an unobserved negative effect), and \textsc{childsnack} (for an unobserved positive effect). \samshort achieved same effects recall of \offlam{} and \nolam, except for \textsc{tpp}, where it only achieved 50\% recall for both positive and negative effects, due to its inability to learn some actions entirely, i.e., the actions have all the preconditions and no effects. 
\rosame achieved the worst effects recall in half of the domains, due to the absence of some positive and negative effects in the learned domains.
Similar considerations apply to preconditions recall, where it is worth noting that \offlam and \nolam in domain \textsc{childsnack} do not learn an action precondition due to the presence of a constant not bound to the action parameters (which is not supported by these algorithms). 
% In terms of effects, both \offlam{} and \nolam obtained syntactic precision and recall of $1$ in all domains except \textsc{rovers} (because of inconsistent effects), \textsc{goldminer} (due to an unobserved negative effect), and \textsc{childsnack} (for an unobserved positive effect). \samshort{} achieved effects the same results except for \textsc{tpp}, where it only achieved 50\% recall.
% for both positive and negative effects.} do we separate the two types in our evaluation?
% However, the domains used in the experiments do not include negative preconditions. 
% This is due to additional preconditions, as both algorithms include negative preconditions in the learned domains, which are always absent in the reference domains. For instance, in the \textsc{blocksworld} domain, the lifted action \textsc{pick\_up}$(x)$ for a block $x$ has the precondition $\textit{handempty}()$, which implies $\textit{holding}(x)$ is false. 
% % However, the domains learned by \samshort{} and \nolam{} include such negative preconditions, which lowers their syntactic precision. 
% In contrast, \offlam{} assumes domains do not include negative preconditions, resulting in more concise domains and higher precision. However, this metric's sensitivity to such additions is a limitation, as it can penalize the inclusion of predicates that implicitly hold true in the domain. 
% Regarding the aggragated recall \roni{recall of preconditions or effects?} \argaman{The metric is calculated as a combination of the two.}, all algorithms performed similarly, with perfect recall in all but four domains -- \textsc{tpp}, \textsc{childsnack}, \textsc{goldminer} and \textsc{rovers}.
% The lower recall in those domains mainly stem from missing effects.
% A prominent example is \textsc{tpp} domain, with \samshort{} having the lowest recall (i.e., $0.5$). 
% This is due to its assumption that actions cannot be executed with the same object bound to multiple parameters—an assumption that does not hold in this domain.\footnote{This is referred to as the \textit{injective binding assumption} in the \sam learning paper.} 
% The ESAM algorithm remedies this limitation~\citep{juba2021safe}.
% % In \textsc{childsnack} domain, \offlam and \nolam both 
% \roni{This explains why SAM did not work well, but what about the other algorithms? what is unique in these domains. If we do not know, that is also Ok, this is not a paper that is intended to explain the algorithms just show the benchmark}
% \leo{In \textsc{childsnack}, both \offlam{} and \nolam{} misses a positive precondition of action $put\_on\_tray$ involving a constant $kitchen$, since they do not support constants. On the opposite, \samshort{} do support constants. In \textsc{rovers}, all approaches cannot learn some positive and negative effects; indeed the IPC version of \textsc{rover} contains some inconsistent effects. In \textsc{gold-miner}, all approaches do not learn a negative effect because it is never observed in the trajectories.}



\miniparagraph{Predicted applicability} 
\samshort and \offlam initially assume the preconditions of each lifted action to be the set of all possible preconditions, and remove unnecessary preconditions that are not consistent with the input trajectories, resulting in perfect applicability precision. Notably, \nolam{} achieves perfect applicability precision even without such an assumption.
The only outlier for \nolam{} and \offlam{} is the \textsc{childsnack} domain, where there is a slight precision decrease due to a missing precondition involving a domain constant.
\rosame provide the worst applicability precision, this is not surprising since the preconditions syntactic recall is lower for \rosame than for other algorithms.
%
In domain \textsc{matchingbw}, the applicability precision achieved by \rosame equals $1$, despite the preconditions syntactic recall equal to $88\%$; similarly for \samshort in domain \textsc{childsnack} and \nolam in domain \textsc{tpp}. These results provide empirical evidence of a syntactic recall drawback previously discussed; indeed, the applicability of an action is not necessarily affected by a reference domain precondition missing in the learned domain, as, e.g., when the missing precondition is logically implied by another precondition in the learned domain. 

% in the precision due to $put\_on\_tray$'s precondition involving $kitchen$.
% This results from the domain including a constant -- $kitchen$ -- in a predicate of the action $put\_on\_tray$'s preconditions. 
% Indeed, this constant is not part of the action's parameters, which causes it not to be learned.
% \roni{Not clear. Do you mean NOLAM and Offlam do not support constants and SAM does?}\leo{Yes, I added a comment about this in the previous paragraph}

% \offlam{} achieved the best performance in terms of applicability recall with perfect recall values in all domains except \textsc{barman}, which had a slight decrease in its recall due to a single literal miss-classified as a precondition. 
\offlam achieved the best performance in terms of applicability recall, with $100\%$ recall in all domains except \textsc{barman} and \textsc{matchingbw}, due to $1$ and $4$ unnecessary (positive) preconditions, respectively. 
% \leo{with only a slight decrease in its recall value due to a false positive precondition for action $empty\_shot$. This results from \offlam{} assuming no negative preconditions, which prevents it from learning unnecessary ones.}
% allows to remove redundant ones with less observations than \samshort{} and \nolam{} that assume the existence of negative preconditions as well. 
%
Notably, in $14$ out of $20$ domains, the applicability recall achieved by \offlam equals $1$ despite its positive preconditions precision being lower than $1$; similarly for \samshort, \rosame, and \nolam in $5$, $10$ and $5$ domains, respectively. 
% Notably, the unnecessary negative preconditions learned by \samshort{} and \nolam{} only significantly impacted the recall for the \textsc{tpp} domain; indeed, all other domains had applicability recall values exceeding $0.6$. 
Moreover, the unnecessary negative preconditions learned by \samshort and \nolam have no impact on the applicability recall in $6$ out of $20$ domains. 
These results emphasize a limitation of the syntactic precision: an unnecessary precondition in the learned domain might not affect the capability of the learned domain of predicting when the action is applicable. 
This is the case when, e.g., the misclassified positive (negative) precondition is a static predicate that is always true (false) in a state where the action is actually applicable in the real environment.


\miniparagraph{Predicted effects} 
All algorithms except \rosame achieved perfect precision and recall across all domains. 
By design, \samshort and \offlam provides effects syntactic precision equal to $1$ (which also holds for \nolam as the trajectories are not noisy), since an action effect is learned only when a literal becomes true/false after observing such action execution in some trajectory transition. This prevents \samshort and \offlam from learning unnecessary effects, thus making the false positives for the predicted effects equal to $0$ and the precision equal to $1$. 

Furthermore, the predicted effects only consider actions applicable according to the learned domain and the real environment, and every lifted action execution is observed at least once in the set of trajectories. 
Therefore, evaluating the actions is only performed when the actions are applicable according to the learned domain. Thus, resulting in states that were already observed in the train trajectories.
Other, less conservative approaches may follow different effect-learning strategies, potentially leading to different precision and recall outcomes; \rosame is an example of such approaches.
Notably, in domains \textsc{floortile} and \textsc{parking}, the predicted effects precision is lower than $1$, despite the syntactic precision being $1$ for both positive and negative effects; this is due to an action with empty preconditions and effects. Note the effects syntactic precision can be misleading, as for the domains learned in \textsc{floortile} and \textsc{parking}, where it overestimates the quality of the learned set of effects, while this is not the case for the predicted effects precision.
\leo{comment rosame matchingbw syntactic precision lower than 1 and predicted effs precision equal to 1. TODO: Leonardo}



\miniparagraph{Problem-solving}  
\offlam{} achieved the highest solving ratio across all domains except for \textsc{childsnack} and \textsc{goldminer}. Notably, in domain \textsc{tpp}, it is the only algorithm to produce a model capable of solving the entire test set of problems.
% In \textsc{elevators} and \textsc{rovers} domains, it highly outperforms both \samshort{} and \nolam{} with 85\% and 70\% increase in the 
% solving rates respectively.
The difficulty in solving problems for \samshort{} and \nolam{} is mainly due to the learned domains having unnecessary negative preconditions. 
%
As expected, the plans produced with the domains learned by \samshort are never false plans; this is guaranteed by the \samshort safety property~\cite{juba2021safe}. 
% Remarkably, the plans produced with the domains learned by \samshort are never false plans; indeed, the safety property of \samshort guarantees the plans computed with the learned domains to be valid plans. \leo{@SAM authors, you may want to double check the previous sentence}
% Specifically, in these domains, the actions had high arity which resulted in many negative preconditions that were created for them (see Table~\ref{tab:domains} for reference). 
In the \textsc{childsnack} domain, both \offlam and \nolam produced inapplicable plans (i.e., the false plans ratio equals $1$) due to a missing precondition involving the unsupported domain constant.
% of $put\_on\_tray$ involving the unsupported constant $kitchen$. 
% This results from the domain having a constant that was not properly learned by these algorithm which was part of the action $put\_on\_tray$'s preconditions.
% An exception is the \textsc{childsnack} domain, which includes a constant unsupported by both NOLAM and OffLAM. 
% \roni{Repetition of the sentence above?}\leo{yes, removed}
% Although plans were generated, they were deemed inapplicable, as shown in Figure~\ref{fig:false-positive-plans}.
% Similarly, the plans produced with the models learned by \offlam{} are inapplicable due to a missing negative effect, which affects the precondition of subsequent plan actions.
In the \textsc{goldminer} domain, the plans produced with the models learned by \offlam{} are inapplicable due to a missing negative effect that was never observed in the input trajectories, which affects the preconditions of subsequent plan actions.
% $(not\; (gold\_at\; ?y))$ of action $fire\_laser$, which affects the precondition of subsequent plan actions.
% We also observed that in the \textsc{goldminer} all of the plans generated using the domains \offlam learned were deemed inapplicable. This resulted from \offlam not learning the effect $(not\; (gold\_at\; ?y))$ which was crucial to the consistency of the resulting plans. 
% Because 
% Such an effect was never observed in the input trajectories. 
Interestingly, \samshort and \nolam, learned the unobserved negative effect as a negative precondition, which allows computing valid plans.
%
It is worth noting that, the \samshort solving ratio in domains \textsc{goldminer} and \textsc{childsnack} equals $1$ despite the syntactic precision of $0.39$ and $0.63$, respectively, and the syntactic recall being lower than $1$; similarly for \nolam in domain \textsc{goldminer} and \offlam in domain \textsc{rovers}. 
Moreover, the solving ratio of \offlam is $0$ in domain \textsc{childsnack} despite the syntactic precision and recall of  $0.97$ and $0.86$, respectively; similarly for \rosame in domains \textsc{barman}, \textsc{childsnack}, \textsc{ferry}, and \textsc{floortile}. 
%
These results provide clear evidence of the syntactic similarity drawbacks previously discussed.
% the syntactic precision performance gap between \offlam{} and other approaches (Figure \ref{fig:syn-precision}) is not reflected in terms of problem-solving (Table \ref{tab:solving_ratio_false_plans_ratio}) as, e.g., in domain \textsc{matchingbw}. 
% This provides empirical evidence of the drawbacks in syntactic similarity discussed in Section \ref{sec:problem-setting}.


% \roni{Important: Can we add to the domains table the columns: Constants (Const.), Injective binding (Inj. Bind), and Negative preconditions (Neg. Pre.)? this will connect to the reults we observed.}
% \leo{added. I am not sure about the injective binding assumption since even though an action can take as input objects of the same type, then the fact that it (can) actually involves the same object multiple times depends on the domain and traces. I am not sure this is easily deducible from the action signature only, as I currently did}


% \roni{The results are Ok. Perhaps would be good to add a summary paragraph that says something like As the results show, different algorithms work well in different domains and for different metrics. }
% \leo{I wrote a draft for this }

% \leo{

\miniparagraph{Results summary} 
Our results show that different algorithms perform better in different domains and for different metrics. 
For example, \sam provides the best applicability precision and false plans ratio, while \offlam achieves the best applicability recall and solving ratio, and \sam, \offlam, and \nolam provide the same performance in terms of predicted effects precision and recall. 


\commentout{
\roni{Commented this out because we have space issue and all metrics are one.}



\begin{table}[b!]
\resizebox{\columnwidth}{!}{
\addtolength{\tabcolsep}{-0.4em}
\begin{tabular}{|l|l|l|l|l|l|l|}
% \toprule
\hline
Domain & \multicolumn{2}{c|}{Applicability} & \multicolumn{2}{c|}{Pred. effs.} & Solv. \% $\uparrow$ & False \% $\downarrow$ \\
 & \makecell[c]{P$\uparrow$} & \makecell[c]{R$\uparrow$} & \makecell[c]{P$\uparrow$} & \makecell[c]{R$\uparrow$} &  & \\
% \midrule
\hline
barman &1.00$^{1,2,4}$ & 0.94 $^{2,3}$ &1.00$^{1,2,4}$ &1.00$^{1,2,4}$ &1.00$^{2}$ &0.00$^{1,2,3,4}$ \\
blocksworld &1.00$^{1,2,4}$ &1.00$^{1,2,3,4}$ &1.00$^{1,2,3,4}$ &1.00$^{1,2,4}$ &1.00$^{1,2,4}$ & 1.00 $^{3}$ \\
childsnack &1.00$^{1}$ &1.00$^{2,3}$ &1.00$^{1,2,3,4}$ &1.00$^{1,2,3,4}$ &1.00$^{1}$ & 1.00 $^{2,3,4}$ \\
depots &1.00$^{1,2,3,4}$ &1.00$^{2,3}$ &1.00$^{1,2,3,4}$ &1.00$^{1,2,3,4}$ &1.00$^{1,2,3,4}$ &0.00$^{1,2,3,4}$ \\
elevators &1.00$^{1,2,3,4}$ &1.00$^{2}$ &1.00$^{1,2,3,4}$ &1.00$^{1,2,3,4}$ &1.00$^{2}$ &0.00$^{1,2,3,4}$ \\
ferry &1.00$^{1,2,4}$ &1.00$^{1,2,3,4}$ &1.00$^{1,2,3,4}$ &1.00$^{1,2,4}$ &1.00$^{1,2,4}$ & 1.00 $^{3}$ \\
floortile &1.00$^{1,2,4}$ &1.00$^{2,3}$ &1.00$^{1,2,4}$ &1.00$^{1,2,4}$ &1.00$^{2}$ &0.00$^{1,2,3,4}$ \\
goldminer &1.00$^{1,2,4}$ &1.00$^{2}$ &1.00$^{1,2,4}$ &1.00$^{1,2,4}$ &1.00$^{1,4}$ & 1.00 $^{2}$ \\
grippers &1.00$^{1,2,3,4}$ &1.00$^{2,3}$ &1.00$^{1,2,3,4}$ &1.00$^{1,2,3,4}$ &1.00$^{1,2,3,4}$ &0.00$^{1,2,3,4}$ \\
matchingbw &1.00$^{1,2,3,4}$ & 0.98 $^{1,2,4}$ &1.00$^{1,2,3,4}$ &1.00$^{1,2,4}$ &1.00$^{1,2,4}$ &0.00$^{1,2,3,4}$ \\
miconic &1.00$^{1,2,3,4}$ &1.00$^{1,2,3,4}$ &1.00$^{1,2,3,4}$ &1.00$^{1,2,3,4}$ &1.00$^{1,2,3,4}$ &0.00$^{1,2,3,4}$ \\
nomystery &1.00$^{1,2,3,4}$ &1.00$^{2,3}$ &1.00$^{1,2,3,4}$ &1.00$^{1,2,3,4}$ &1.00$^{2,3}$ &0.00$^{1,2,3,4}$ \\
npuzzle &1.00$^{1,2,3,4}$ &1.00$^{1,2,3,4}$ &1.00$^{1,2,3,4}$ &1.00$^{1,2,3,4}$ &1.00$^{1,2,3,4}$ &0.00$^{1,2,3,4}$ \\
parking &1.00$^{1,2,4}$ &1.00$^{2,3}$ &1.00$^{1,2,4}$ &1.00$^{1,2,4}$ &1.00$^{2}$ &0.00$^{1,2,3,4}$ \\
rovers &1.00$^{1,2,4}$ &1.00$^{2}$ &1.00$^{1,2,4}$ &1.00$^{1,2,4}$ &1.00$^{2}$ &0.00$^{1,2,3,4}$ \\
satellite &1.00$^{1,2,3,4}$ &1.00$^{2,3}$ &1.00$^{1,2,3,4}$ &1.00$^{1,2,3,4}$ &1.00$^{1,2,3,4}$ &0.00$^{1,2,3,4}$ \\
sokoban &1.00$^{1,2,3,4}$ &1.00$^{1,2,3,4}$ &1.00$^{1,2,3,4}$ &1.00$^{1,2,3,4}$ &1.00$^{1,2,4}$ &0.00$^{1,2,3,4}$ \\
spanner &1.00$^{1,2,3,4}$ &1.00$^{1,2,3,4}$ &1.00$^{1,2,3,4}$ &1.00$^{1,2,3,4}$ &1.00$^{1,2,3,4}$ &0.00$^{1,2,3,4}$ \\
tpp &1.00$^{1,2,3,4}$ &1.00$^{2}$ &1.00$^{1,2,3,4}$ &1.00$^{1,2,3,4}$ &1.00$^{2}$ &0.00$^{1,2,3,4}$ \\
transport &1.00$^{1,2,3,4}$ &1.00$^{1,2,3,4}$ &1.00$^{1,2,3,4}$ &1.00$^{1,2,3,4}$ &1.00$^{1,2,3,4}$ &0.00$^{1,2,3,4}$ \\
% \bottomrule
\hline
\end{tabular}
}
\caption{\roni{Need to update this given the new results by ROSAME.}\leo{updated}\roni{The False\% results are weird: it seems the best algorithm in blocksworld, ferry, and goldminer ahve 100\% false plans.That does not seem right. Are you sure?} Best metric values for every domain obtained among \sam, \offlam, \rosame, \nolam, respectively denoted by 1,2,3 and 4; P indicates the precision and R the recall, $\uparrow$ (resp. $\downarrow$) denotes the higher (resp. lower) the better. 
% The predicted effects precision and recall equal $1$ for all approaches in all domains.
}
\label{tab:best}
\end{table}




% \hline
% barman  & 0.95 $^{2}$ & 1.00 $^{1,2,3}$ & 1 $^{1,2,3}$ & 0.94 $^{2}$ & 1 $^{2}$ &0.00$^{1,2,3}$ \\
% blocksworld  & 1 $^{2}$ & 1 $^{1,2,3}$ & 1 $^{1,2,3}$ & 1 $^{1,2,3}$ & 1 $^{1,2,3}$ &0.00$^{1,2,3}$ \\
% childsnack & 0.97 $^{2}$ & 0.9 $^{1}$ & 1 $^{1}$ & 1 $^{2}$ & 1 $^{1}$ &0.00$^{1}$ \\
% depots  & 0.98 $^{2}$ & 1 $^{1,2,3}$ & 1 $^{1,2,3}$ & 1 $^{2}$ & 1 $^{1,2,3}$ &0.00$^{1,2,3}$ \\
% elevators  & 0.81 $^{2}$ & 1 $^{1,2,3}$ & 1 $^{1,2,3}$ & 1 $^{2}$ & 1 $^{2}$ &0.00$^{1,2,3}$ \\
% ferry  & 0.93 $^{2}$ & 1 $^{1,2,3}$ & 1 $^{1,2,3}$ & 1 $^{1,2,3}$ & 1 $^{1,2,3}$ &0.00$^{1,2,3}$ \\
% floortile & 0.83 $^{2}$ & 1 $^{1,2,3}$ & 1 $^{1,2,3}$ & 1 $^{2}$ & 1 $^{2}$ &0.00$^{1,2,3}$ \\
% goldminer & 0.8 $^{2}$ & 0.98 $^{1,2,3}$ & 1 $^{1,2,3}$ & 1 $^{2}$ & 1 $^{1,3}$ &0.00$^{1,3}$ \\
% grippers & 1 $^{2}$ & 1 $^{1,2,3}$ & 1 $^{1,2,3}$ & 1 $^{2}$ & 1 $^{1,2,3}$ &0.00$^{1,2,3}$ \\
% matchingbw & 0.95 $^{2}$ & 1 $^{1,2,3}$ & 1 $^{1,2,3}$ & 0.98 $^{1,2,3}$ & 1 $^{1,2,3}$ &0.00$^{1,2,3}$ \\
% miconic & 1 $^{2}$ & 1 $^{1,2,3}$ & 1 $^{1,2,3}$ & 1 $^{1,2,3}$ & 1 $^{1,2,3}$ &0.00$^{1,2,3}$ \\
% nomystery & 0.94 $^{2}$ & 1 $^{1,2,3}$ & 1 $^{1,2,3}$ & 1 $^{2}$ & 1 $^{2}$ &0.00$^{1,2,3}$ \\
% npuzzle & 0.88 $^{2}$ & 1 $^{1,2,3}$ & 1 $^{1,2,3}$ & 1 $^{1,2,3}$ & 1 $^{1,2,3}$ &0.00$^{1,2,3}$ \\
% parking & 0.89 $^{2}$ & 1 $^{1,2,3}$ & 1 $^{1,2,3}$ & 1 $^{2}$ & 1 $^{2}$ &0.00$^{1,2,3}$ \\
% rovers & 0.83 $^{2}$ & 0.93 $^{1,2,3}$ & 1 $^{1,2,3}$ & 1 $^{2}$ & 1 $^{2}$ &0.00$^{1,2,3}$ \\
% satellite & 1 $^{2}$ & 1 $^{1,2,3}$ & 1 $^{1,2,3}$ & 1 $^{2}$ & 1 $^{1,2,3}$ &0.00$^{1,2,3}$ \\
% sokoban & 0.88 $^{2}$ & 1 $^{1,2,3}$ & 1 $^{1,2,3}$ & 1 $^{1,2,3}$ & 1 $^{1,2,3}$ &0.00$^{1,2,3}$ \\
% spanner & 0.93 $^{2}$ & 1 $^{1,2,3}$ & 1 $^{1,2,3}$ & 1 $^{1,2,3}$ & 1 $^{1,2,3}$ &0.00$^{1,2,3}$ \\
% tpp & 0.95 $^{2}$ & 1 $^{2}$ & 1 $^{1,2,3}$ & 1 $^{2}$ & 1 $^{2}$ &0.00$^{1,2,3}$ \\
% transport & 0.93 $^{2}$ & 1 $^{1,2,3}$ & 1 $^{1,2,3}$ & 1 $^{1,2,3}$ & 1 $^{1,2,3}$ &0.00$^{1,2,3}$ \\
% % \bottomrule
% \hline

\miniparagraph{Results summary} 
Table \ref{tab:best} lists the best performance for each proposed metric and domain on the proposed benchmark. The superscripts in each cell indicate which of the evaluated algorithms --- \sam, \offlam, \rosame, and \nolam --- obtained the best performance in this metric and domain. As the results show, while the algorithms can achieve impressive performance in our benchmark, it can be noticed that different algorithms perform better in different domains and for different metrics. 
For example, \sam provides the best applicability precision and false plans ratio, while \offlam achieves the best applicability recall and solving ratio, and \sam, \offlam, and \nolam provide the same performance in terms of predicted effects precision and recall. 
% better syntactic precondition precision than \samshort and \nolam in all domains. However, it fails to solve some test problems in \textsc{goldminer} since it does not learn negative effects (while \sam and \offlam can learn them). 
% are not learned, as in \textsc{goldminer} domain; while this is not the case for \samshort{} and \nolam{}. 
% Indeed, the syntactic precision performance gap between \offlam{} and other approaches (Figure \ref{fig:syn-precision}) is not reflected in terms of problem-solving (Figure \ref{fig:solving-ratio}) as, e.g., in domain \textsc{matchingbw}. This provides empirical evidence of the drawbacks in syntactic similarity discussed in Section \ref{sec:problem-setting}.
Interestingly, the different approaches to deal with action objects binding can significantly impact the performance in domains where the injective action binding assumption is violated, as in \textsc{elevators} and \textsc{tpp}.  
}

% BACKUP IF I DO DAMAGE
% We assessed the best performance of three state-of-the-art approaches on the proposed benchmark (Table \ref{tab:best}). As the results show, different algorithms perform better in different domains and for different metrics. On the one hand, \offlam{} provides better syntactic precision than \samshort{} and \nolam{}, on the other hand, it assumes there are no negative preconditions, which can degrade problem-solvability performance when some negative effects are not learned, as in \textsc{goldminer} domain; while this is not the case for \samshort{} and \nolam{}. 
% Moreover, the syntactic precision performance gap between \offlam{} and other approaches (Figure \ref{fig:syn-precision}) is not reflected in terms of problem-solving (Figure \ref{fig:solving-ratio}) as, e.g., in domain \textsc{matchingbw}. This provides empirical evidence of the drawbacks in syntactic similarity discussed in Section \ref{sec:problem-setting}.
% Interestingly, the different approaches to deal with action objects binding can significantly impact the performance in domains where the injective action binding assumption is violated, as in \textsc{elevators} and \textsc{tpp}.  


\section{Discussion}
We explored three families of metrics: syntactic similarity, predictive power, and problem-solving ability. 
There is an interesting trade-off between these families of metrics. 
The syntactic similarity is the easiest to compute, as it only requires comparing the learned domain with a reference domain. However, they are arguably the least useful, and computing them requires many assumptions (Assumptions A1 and A2 above). 
The predictive power metrics characterize well a desirable property of a domain model, and are still relatively easy to compute. 
The downside of these metrics is that in contrast to classical machine learning settings, the ``data distribution'' in planning is highly \emph{non-stationary} (not ``i.i.d.''):  Although the states in the benchmarks were selected to be representative of those visited on trajectories collected using the reference model, this may change. Indeed, the states encountered on a trajectory generated by a planner using a given learned domain-model representation depend on the model. Planners can exploit deficiencies in the learned representation to systematically visit states that are erroneously modeled, if those errors make goals easier to achieve. Thus, the predictive power metrics can be a poor proxy for the utility of the learned model for planning. The problem-based metrics fill in this gap, but, of course, they are the hardest to compute as they require running a planner to compute them. 



% providing a clear, well-defined benchmark for estimating the predictive power of method


% and are still 
% easy to compute and thus provide a clear, well-defined benchmark for estimating the predictive power of methods. This is in contrast to the problem-based metrics that rely on the use of planners that are compute-intensive and, depending on the resources available, may have rather different success rates for the same model representation. The downside of state-based metrics is that in contrast to classical machine learning settings, the ``data distribution'' in planning is highly \emph{non-stationary} (not ``i.i.d.''):  Although the states in the benchmarks were selected to be representative of those visited on trajectories collected using the reference model, this may change. Indeed, the states encountered on a trajectory generated by a planner using a given learned domain-model representation depend on the model. Planners can exploit deficiencies in the learned representation to systematically visit states that are erroneously modeled, if those errors make goals easier to achieve. Thus, the state-based metrics may sometimes be a poor proxy for the utility of the learned model for planning, and here, the problem-based metrics are crucial.}
% \roni{This is super. I'll incorporate it in.}




\section{Conclusion and Future Work}
We proposed a comprehensive evaluation process and a suite of metrics for assessing domain model learning algorithms. Our framework includes three complementary families of metrics: syntactic similarity, predictive power, and problem-solving ability. We analyzed the strengths and limitations of each metric family and how they jointly capture different aspects of model quality.
We also describe our implementation of an evaluation process that computes these metrics, along with a benchmark of domain model learning problems that builds on domains from the International Planning Competition (IPC) learning track.
Empirical evaluation of three state-of-the-art domain model learning algorithms on this benchmark reveals that each algorithm exhibits distinct strengths and weaknesses.
% To address the challenge of comparing learned domain models expressed in different representations, we further proposed an encoder-decoder mechanism that enables meaningful comparison and metrics across heterogeneous outputs.
The proposed benchmark and evaluation framework are publicly available, providing a foundation for systematic and reproducible research in this area.
Future work may extend the benchmark to include domains beyond the IPC and adapt the evaluation metrics to settings involving online or incremental learning~\citep{lamanna2021online, sreedharan2023optimistic, benyamin2025integratingreinforcementlearningaction, ng2019incremental, chitnis2021glib, jin2022creativity, verma2023autonomous, karia2023epistemic}. 
In addition, we intend to develop an evaluation process and metrics for cases where the state representation used by the different models are different, as well as non-classical planning domain model learning algorithms~\cite{mordoch2023learning} and online learning settings~\cite{lamanna2021online}.


\commentout{
\miniparagraph{Online learning of domain models}
Here, the learning algorithm is required to learn a domain model by actively interacting with the environment~\citep{lamanna2021online, sreedharan2023optimistic, benyamin2025integratingreinforcementlearningaction, ng2019incremental, chitnis2021glib, jin2022creativity, verma2023autonomous, karia2023epistemic}. 
All the proposed metrics can be computed as-is, after a predefined number of iterations, for the online learning setting. 
However, an additional perspective that may be of interest in online learning of domain models is \emph{cumulative} metrics of the learning process, such as the number of actions performed for learning or the number of problems failed to solve while collecting observations.
% \brendan{Changed terminology. This has little to do with ``regret,'' which refers to fixing, e.g., a single best policy from a class in hindsight as a baseline for comparison.}  Sounds good.
Specific metrics and evaluation for online learning of domain models, however, are beyond the scope of this work. 
}



% we proposed an evaluation process and metrics for domain model learning algorithms. 
% This includes three families of metrics, syntactic similarity, predictive power, and problem-solving ability, and discussed their complementary strengths and weaknesses. 
% % of these metrics and propose an evaluation process for computing them. 
% Then, we described our implementation of this evaluation process and a set benchmark problems based on domains from the IPC. 
% Empirical results on three state of the art domain model learning algorithms over this benchmark show that each algorithm has its pros and cons. 
% Finally, we proposed an encoder-decoder mechanism for comparing domain model learning algorithms that output domains that use different representation. 
% The benchmark and evaluation process we proposed are publicly available, facilitating future research in the field to progress. Future work can also add to our benchmark domains beyond the IPC, as well as extend our metrics to the online learning setting. 


% \section{Conclusion Future Work}
% \roni{I'll fix this text later}
% % TODO VERIFY TEXT BELOW
% We presented a new paradigm for evaluating action 
% model learning algorithms across different representations. In this paradigm, each domain model learning algorithm is required to output a domain model and an encoder-decoder pair. The encoder-decoder pair is used to bridge representation gaps and enable measuring the problem-solving capabilities of the learned domain models. We proposed an evaluation scheme that leverages the encoder-decoder pair to systematically compare learned domain models and described several evaluation metrics. A benchmark suite was also provided to facilitate the evaluation of domain model learning algorithms, based on existing domain models from the International Planning Competition (IPC). We demonstrated our evaluation paradigm by applying it to several domain model learning algorithms, including \sam, ESAM, FAMA, and ROSAME.

% Finally, domain models can also be compared not only in terms of their characteristics, but also in terms of the processes used to learn or generate them \citep{vallati2021quality}, a perspective that is commonly used for assessing conceptual models. This is also a direction for future work. 


\bibliography{library} 
\end{document}
